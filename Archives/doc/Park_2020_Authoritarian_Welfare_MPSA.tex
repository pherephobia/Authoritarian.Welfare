\documentclass[11pt, notitlepage]{article}
\usepackage{setspace}
\doublespacing
\usepackage{geometry}
\usepackage{titling}
\usepackage{blindtext}
\geometry{margin=1in}
\usepackage{graphics} % for pdf, bitmapped graphics files
\usepackage{epsfig} % for postscript graphics files
\usepackage{mathptmx} % assumes new font selection scheme installed
\usepackage{times} % assumes new font selection scheme installed
\usepackage[fleqn]{amsmath} % assumes amsmath package installed
\usepackage{amssymb}  % assumes amsmath package installed
\usepackage[affil-it]{authblk}
\usepackage{natbib}
\bibpunct{(}{)}{;}{a}{}{,}
\usepackage{hyperref}
\usepackage{bookmark}
\hypersetup{
	colorlinks   = true, %Colours links instead of ugly boxes
	urlcolor     = blue, %Colour for external hyperlinks
	linkcolor    = blue, %Colour of internal links
	citecolor   = blue %Colour of citations,
}
\usepackage{xcolor}
%\usepackage{lettrine}
\usepackage{longtable}
\usepackage{afterpage}
\usepackage{booktabs}
\usepackage[utf8]{inputenc}
\usepackage[sc]{mathpazo}
\usepackage[T1]{fontenc}
\usepackage{adjustbox}
\usepackage{ntheorem}
\theoremseparator{:}
\newtheorem{hyp}{Hypothesis}
\usepackage[toc,title,page]{appendix}

\makeatletter % <=======================================================
\renewcommand\@seccntformat[1]{}
\renewcommand{\@makefntext}[1]{%
	\setlength{\parindent}{0pt}%
	\begin{list}{}{\setlength{\labelwidth}{6mm}% 1.5em <==================
			\setlength{\leftmargin}{\labelwidth}%
			\setlength{\labelsep}{3pt}%
			\setlength{\itemsep}{0pt}%
			\setlength{\parsep}{0pt}%
			\setlength{\topsep}{0pt}%
			\footnotesize}%
		\item[\@thefnmark\hfil]#1% @makefnmark
	\end{list}%
}

\makeatother % <========================================================


\title{\bf Why Do Authoritarian Regimes Provide Welfare?}
\author{Sanghoon Park
	\thanks{\small Ph.D. Student, Department of Political Science, University of South Carolina\\
		 \hspace*{1.8em}(\href{sp23@email.sc.edu}{sp23@email.sc.edu})}}
	
\date{\today}


\begin{document}
\begin{titlingpage}
	\maketitle
   \begin{center}
	\centering \onehalfspacing \it Prepared for the 2020 Annual Meeting of\\the Midwest Political Science Association
\end{center}

	
\begin{abstract}
\onehalfspacing
\noindent This paper evaluates why authoritarian regimes design social policies to promote the welfare of citizens. Social welfare policies are coordinated outcomes that link the political and economic realms together. Despite a rich literature on this topic in industrialized democracies, little research has investigated the mechanisms in authoritarian regimes. I argue that the limited focus on welfare provision in democracies stems from several flawed assumptions about the ability of modern autocracies to create similar programs and the mechanisms behind it. Although autocrats are more autonomous decision makers than democratic leaders, they rarely rule alone. As such, social policies in authoritarian regimes are driven by the autocrats’ need to create stable ruling coalitions. The welfare provided in authoritarian regimes should differ depending on the groups on which the autocrat depends for support. The ability to effectively decommodify welfare and provide assistance should also vary based on regime characteristics and resource availability. I examine this using data on authoritarian regimes and social policies over the period 1917 to 2000, showing that authoritarian welfare provision depends on the ruling coalition in place.
% Change the year
\end{abstract}
\end{titlingpage}


\noindent How successfully can we explain the variations of welfare provision among countries with the existing theories? Scholars have investigated the causes and consequences of welfare state variation for several decades, and it has become one of the major themes in comparative politics. Questions of the welfare state, however, have covered the variations of welfare within advanced industrial democracies \citep{Pierson1996, Pierson2000}. Despite a rich literature on this topic in industrialized democracies, little research has investigated the mechanisms in authoritarian regimes. Do authoritarian regimes provide welfare to their citizens? Research on the welfare of democracies expects that democracies would provide greater stability and more effective public policies for citizens to foster their quality of life \citep{Gerring2012}. Authoritarian regimes also design social policies and provide a set of welfare programs, but it is not clear for whom the policies and programs are targeting, unlike democracies. Then, what would be a key mechanism to explain the variation of welfare in the non-democratic world?\footnote{The paper uses the terminologies of non-democracies, such as `authoritarian regimes.' The authoritarian regime, dictatorship, and autocracy interchangeably.}\par

This paper evaluates why authoritarian regimes design social policies to promote the welfare of citizens. Social welfare policies are coordinated outcomes that link the political and economic realms together. Many theoretical studies have proposed that democracy is somehow related to higher levels of welfare provisions \citep{Muller1988, Sirowy1990, Korpi1998} with several key concepts; the logic of industrialism \citep{Heclo1974, Wilensky1975, PrzeworskiandLimongi1993}, inherent characteristics of democratic institutions \citep{Muller1988, Pierson1996, Gerring2012}, compensating for the specific social groups \citep{Cameron1978, Rodrik1998, Burgoon2001a, Adsera2002} and distribution of political resources across classes within a society \citep{Korpi1998, Gosta1989, Gosta1990}. The central mechanism of conventional wisdom explains variations of welfare provision with social pressures for extensive welfare. Since a democratic government is more subject to demands from citizens, in theory, by promoting political equality, democracy provides various groups, such as interest groups, labor unions, and political parties, with open spaces of political competition to represent their interests and welfare. However, very few studies have attempted to explain the variation of welfare with the mechanisms in existing theories for authoritarian welfare states.\par

I argue that the limited focus on welfare provision in democracies stems from several flawed assumptions about the ability of modern autocracies to create similar programs and the mechanisms behind it. Although autocrats are more autonomous decision-makers than democratic leaders, they rarely rule alone. As such, social policies in authoritarian regimes are driven by the autocrats’ need to create stable ruling coalitions. The welfare provided in authoritarian regimes should differ depending on the groups on which the autocrat depends for support. The ability to effectively decommodify welfare and provide assistance should also vary based on regime characteristics and resource availability. I examine this using data on authoritarian regimes and welfare programs, showing that authoritarian welfare provision depends on the ruling coalition in place.\par

I organize the remainder of the paper as follows. In the next two sections, I review the theoretical backgrounds of theories on the welfare state. From the previous explanations of the welfare state, I develop the theoretical framework of our analysis. The following sections describe the dataset and empirical models I have constructed to test the hypotheses generated by this framework. The subsequent sections present and discuss our empirical results, and we conclude by addressing issues for further inquiry.

\section{Literature Reviews} 
\subsection{Understanding Welfare Regimes} 
Studies of welfare state have mostly been conducted in democratic countries with comparable and available data \citep{Jessop1984, Gosta1989, Gosta1990, Rudra2005, Huber2008, Hudson2009}. From the 19th century, as Western democracies expanded the universal franchise, the levels and coverage of social policies had expanded. Previous studies have provided empirical evidence that democracy has seen broader and deeper levels of welfare \citep{Acemoglu2000, Lindert2005}. They have made great efforts to find the regularity of social policies in democracies, and the strands of inquiry are so-called ``Welfare state'' studies.\par

The phrase welfare state was first used to describe Labour Britain after 1945, and the first usage of the phrase rarely defined what it is but just described it with democratic sentiments such as social rights \citep[9-10]{Briggs1961}. The label ‘welfare state’ incorporates not only Britain but also all modern industrialized countries, defining “a state in which organized power is deliberately used to modify the play of market forces in at least three directions---first, by guaranteeing individuals and families a minimum income irrespective of the market value of their work or their property; second, but narrowing the extent of insecurity by enabling individuals and families to meet certain ‘social contingencies’ (for example, sickness, old age, and unemployment) which lead otherwise to individual and family crises; and third, by ensuring that all citizens without distinction of status or class are offered the best standards available in relation to a certain agreed range of social services” \citep[14]{Briggs1961} Based on the definition, several studies have investigated what makes differences in social expenditure across countries \citep{Cameron1978, Rodrik1998, Adsera2002}.\par

However, the prominent study on welfare state, \citet{Gosta1990} argues that we should not focus on particular social policy \textit{per se}, but on how different countries arrive at their peculiar public-private sector mix \citep[2]{Gosta1990}. According to \citet{Gosta1990}, tracing the development of a welfare state with the size of social expenditure (or public spending) may be ahistorical and overlook the process. \citet{Gosta1990} focused on the two significant dimensions of welfare, decommodification, and stratification. On the one hand, decommodification is an indicator as to where the status of individuals in the market is. A state can intervene to secure the lost in the market through compensating for potential loss of job, income. Also, a state can provide general welfare for protecting citizens from sickness, injuries, unemployment. On the other hand, stratification is the status of individuals based on their class positions \citep{Gosta1990}.\par

The fundamental mechanism of the research is based on a class coalition. With qualitative research, \citet{Gosta1990} suggests that the structure of class coalitions is vital, pressing the ruling government toward specific types of welfare states. Thus, it is essential which classes go to make a coalition and whether the coalition successfully takes the ruling initiative of the state. It implies that the attempt of \citet{Gosta1990} to develop a typology of welfare states is based on power distribution in a society. He is, therefore, centrally concerned with how to explain the power structures underlying different welfare states \citep[88]{Kemeny1995}. In other words, the three types of the welfare state in \citet{Gosta1990}---Liberal, Conservative, and Social democracy represent macro-features of how a country forms the class coalition to design and provides welfare \citep[646]{Scruggs2008}.

Subsequent studies of \citet{Gosta1990} continued to categorize the welfare state based on different characteristics \citep[14]{Arcanjo2006}; coverage, citizenship right \citep{Ferrera1996}, relative size of social expenditure as a percentage of gross domestic products \citep{Bonoli1997}, or needs and contributions \citep{Korpi1998}. The main argument of \citet{Gosta1990} and following studies is that differential structuring of power relationships between classes is a political and policy-making process, making it possible to distinguish between different types of the welfare state. In other words, the thesis of \citet{Gosta1990} is much more than a descriptive typology but is ultimately founded on power relationships between social classes \citep[89]{Kemeny1995}.\par

\subsection{Welfare Regimes in Autocracies}

Some criticize that the \textit{three worlds of welfare state} are only tested in limited democracies and not eligible to explain the variations of welfare states, which are on the boundary between the types. It is partially right. We cannot solve the puzzle of welfare state concerning authoritarian regimes with the existing theories on the welfare state since the previous arguments assume particular values of democratic institutions. When a state is an authoritarian regime, it does not guarantee compensation its citizens for the losses from the market. In authoritarian regimes, the power of the state is more likely personalized, and the rules are more likely to be established simply and efficiently using fiat and forces \citep[4]{Gerring2012}. However, the mechanism of \citet{Gosta1990}'s three worlds are somehow related to power resource theory, and it may be the key to solve the puzzles in authoritarian regimes with `class coalition.' \par

The argument of power resource suggests that democracies provide welfare to its citizens because they rely on broader coalitions. If so, on the one hand, the fundamental argument of power resource is nothing more than the Selectorate theory \citep{BDM1999, BuenodeMesquita2003}. The selectorate theory states that leaders, facing challengers who wish to depose them, maintain their coalitions of supporters by taxing and spending in ways that allocate mixtures of public and private goods. The theory assumes each member of the selectorate (\textit{S}) who is responsible for making a leader has an equal probability of being in a  winning coalition. According to this logic, as the size of the winning coalition (\textit{W}) becomes smaller or the size of the selectorate becomes larger, challenges are less likely to need or use the support of any particular individual when forming their winning coalition. It is riskier to defect when the size of \textit{W} shrinks or the size of \textit{S} grows.\par 

On the other hand, studies on authoritarian regimes argue that we should consider the \textit{varieties of authoritarian regimes}, which means not every authoritarian regime is the same \citep{Geddes1999, Geddes2014, Cheibub2010, Wahman2013, Roller2013, Soest2015}. In reality, we can observe authoritarian regimes have distinctive leadership groups such as dominant party, military, monarchy, or personalized factions \citep{Geddes2014}. Generally, studies have a consensus that the party based regime has a larger coalition compared to the personalist regime, which has a small coalition to remain in office \citep{Levitsky2002, Levitsky2010a, Gandhi2009}. It is the point where the two lines of inquiry of selectorate theory and varieties of authoritarian regimes meet to explain the authoritarian welfare state.\par

The welfare state literature has revealed how democracies provide welfare to their citizens, and authoritarian regimes provide welfare programs \citep{Tang2000, Kwon2005a, Orenstein2008, Bader2015, Ong2015, Morgenbesser2017}. Regardless of the regime types, political leaders seek to stay in office. It means that autocrats also have incentives to provide public goods to their citizens \citep{Wintrobe1998}. The difference between democracies and non-democracies is that the leaders under non-democracies are less constrained to make decisions. In terms of providing a public good to stay in office, we would expect some non-democracies to be subject to smaller dynamics discussed in the literature of democratization on the welfare regime and democracies. Given the fact that authoritarian regimes can be seen as the diminished form of democracies and given the variations in authoritarian regimes, we can expect that some forms of authoritarian regimes are more likely to provide welfare. In other words, autocrats are more autonomous decision-makers than democratic leaders; However, they rarely rule alone. The needs of autocrats to secure their ruling coalition which is a coordinated social group to secure the political survival of autocrats and their stable ruling drive the social policies to promote welfare even in authoritarian regimes \citep{Svolik2012}. Thus, with smaller coalitions, authoritarian regimes provide welfare because it is closely related to the credible commitment for cooptation.

Explanations of welfare provision in democracies have something in common. They argue that the social groups belonging to selectorate transform their needs into a sort of pressure to push towards particular sets of social policies. For example, a simple but essential analytical model, relating democratic institutions and welfare, comes from \citet{Meltzer1981}, who focus on the effects of electoral competition.\footnote{\citet{Boix2003} provide overviews of this issue} This model assumes that the median voter is the critical voter to determine the size of government, which is measured by the share of income redistributed. It implies that the size of the government depends on the relationship between mean income and median income. Here, the size of selectorate that depends on the regime types is important. Whereas democracy provides a universal franchise, which leads the median voter to have a below-average income in an unequal society, one of the poor who favors higher taxes and more welfare. In authoritarian regimes, the median voter who can affect the policy-making process would belong to a much smaller selectorate. It means that the median voter in an authoritarian regime is more likely to be one of the rich or the upper-class \citep[2]{Yi2014}.\par

The studies on \textit{varieties of authoritarian regimes} have focused on which social groups matter and affect the decision of autocrats. From the \citet{Geddes1999} and \citet{Geddes2014}, scholars have attempted to uncover the decisive leadership group, which controls the policy-making process within the regime. However, when we connect the previous discussion of authoritarian welfare to class coalition---ruling coalition of autocracies, it becomes clear that we need to concern ourselves with how much autocrats consider the selectorate and how large the coalition size is because it shows the extent of welfare the autocrats should provide. It means that the welfare provided in authoritarian regimes should differ depending on the groups on which the autocrat depends for support.\par

Even though some autocrats use forces and coercions to rule the regime, not all do. Autocrats have two fundamental problems, which cannot be solved by a single strategy. One is the problem of power-sharing, which is about challenges from the elites. The other is the problem of control in which autocrats face threats from the masses they rule. The processes to resolve the problems in authoritarian regimes are not transparent and formal, contrary to in democracies. When an autocrat faces these two fundamental problems, she may have two strategies of coercion and cooptation. The autocratic welfare is closely related to the latter strategy. For instance, authoritarian regimes establish institutions to encourage regime stability for autocrat through signaling secured property rights and credible commitment \citep{Gandhi2007, Wright2008}.

\section{Theory}
\subsection{The Primacy of Coalitions}

In authoritarian regimes, we can expect the welfare might be the same as the means to co-opt the classes from selectorate of which the members are potential constituents for the ruling coalition. Given that authoritarian social policies are to effectively stabilize and sustain the system through co-opting the supports of selectorate, social policies should be more credible and reliable than private goods, which the autocrat can distribute at his discretion. \citet{Knutsen2018} emphasizes that pensions are a good example of institutionalized club goods for autocrats to use with particular target groups. They specify old-age pensions as a useful tool for autocrats to reduce the threats of revolt and to co-opt their opponents. They argue that especially old-age pensions affect a wide range of beneficiaries as the pensions do not depend on an unexpected event such as unemployment, sickness. The old-age pensions may be relatively universal to other types of welfare. In the authoritarian regime, as mentioned above, autocrats only have the incentive to provide old-age pension programs to secure targeted supporter groups \citep[670]{Knutsen2018}. In conclusion, \citet[688]{Knutsen2018} argue that ``major pension programs are not restricted to democracies $\cdots$ Autocrats use pensions to direct resources to critical supporting groups.''

How universal welfare programs are offered, and which ones to focus more on, depends on which class the regime should co-opt. This study understands that class exerts social pressure on decision-makers through different and distinctive interests or means of political mobilization. Therefore, it is necessary to examine which classes are politically significant and identifiable \citep{Bean1998}. In this paper, I take the assumption of \citet[1495]{Dahlum2019}: individuals who belong to similar socioeconomic strata are more likely to have converging preferences over social policies.

%In Western Europe, agriculture was the dominant production structure of society until the industrial revolution between the late 18th and early 19th centuries emerged. The source of the political resources of the elite was based on agriculture. Although the landowners of the peasants and bourgeois expanded over time, the feudal manor lords occupied the land ownership steadily. The feudal lords---the agrarian elite has accumulated surplus labor or surplus products from the peasants or tenant farmers in the form of rents by lending common lands. \citet{Ansell2015a} suggest that the ruling elite in the early stage can be divided into two groups. They call the two groups as the relative economic elites and the autocratic elite \footnote{This study defines the relative economic elites as groups of people who have industrial goods (economic power) without political power.}. Also, \citet{Ansell2015a} considers the autocratic elite as the incumbent elite who gains his resources from its local agricultural section.

A line of inquiry argues economic and class-based explanations to account for democratic transition, development \citep[4-5]{ziblatt2017conservative}, and welfare state \citep{Huber1993,HuberStephens2001,Korpi2006}. The explanations usually classify classes that are important in a regime into two types, \textit{middle class} and \textit{working class}. On the one hand, the middle class is the group that has the economic capability. The middle class contributes to economic balance and, at the same time, serves as a buffer against political extremes. Socially, it also serves as a role model for low-income groups, while playing a role of social balance by having a specific collective identity that is distinct from the upper classes. For example, Some evaluate the middle class as the driving force for democratization historically \citep{Moore1993Social}. In line with literature on democratization, we can expect that the middle class can be seen as an effective source of threatening the regime. It also makes the middle class important when I explore the influence of classes on welfare programs in authoritarian regimes.

The middle class can be identified as a politically significant social group to co-opt since whether to make a coalition with the middle class determine the costs of repression. Also, theories that find the cause of economic development in the distribution of political power also suggested that the middle-class influences market growth by contributing to the development of democracy and preventing dictatorship \citep{Easterly2001}. Many studies suggest that middle-class supplies crucial support for the popular mobilizations \citep{Rosenfeld2017, Dahlum2019}. Regarding incentives to co-opt, however, the middle class can be seen as a heterogeneous group. Remaining in office, authoritarian leaders should make a coalition with the group that might have motivation and capability for regime transition \citep{Dahlum2019}. Thus, I expect that the middle class, in particular, the urban middle class, would have higher leverage for authoritarian leaders.

On the other hand, the working class can be an important factor in the development of redistributive social policy, partly through its effects on the partisan composition of government (\citealp{HuberStephens2001}, \citealp[794]{Rasmussen2018}). According to \citet[175]{Korpi2006}, the working class is a collectively mobilized group of employees who are dependent on the labor force. Mobilization of the working class is closely related to the development of the welfare state \citep{Shalev1983}. The expectation that the working class will have a converging preference is based on the assumption that the group of workers has the same interest in ensuring security from the risk of the market. Although \citet[1495]{Dahlum2019} argues that the urban working class (e.g., industrial workers) and the rural working class (peasants) show different levels of capabilities to organize nationwide or localize strikes for the regime transitions, it has different narratives in terms of co-optation. Since the rural working class can be influenced by urbanization or industrialization, the rural working class, and the urban class are not exclusive to each other. \citet[1932]{Chandra2002} shows an example of the Russian revolution. The rural working class is understood as the transitional group between the remnants of feudal and the capitalist mode of production. The rural working class and the urban working class were co-opted because they were in a continuous line by looking at the production of rural working classes as the basis of capitalist development. 

Besides the income based classes, groups based on specific institutions also matter with regards to the political survival of autocrats. \citet{Geddes1999} and \citet{Geddes2014} proposes different types of leadership groups are able to make the most important decisions in the authoritarian regime. This group makes key policies, and autocrats must retain the support of its members to remain in power, even though autocrats may also have substantial ability to influence the group's membership. Institution-based class refers to additional coalition members, and it provides valuable alternative bluntly constructed regime types. Specific institutions create a class of elites with distinct interests and incentives. For instance, parties are not the only way to mobilize the masses, but if created, it may offer goods to its leadership, party elites. 

The military and party elites are the institution-based groups, which show close ties with authoritarian leaders. First, the military does not depend on military leaders’ personal characteristics. Instead, the military is characterized by collegial forms meaning ``rule by the military as institution'' \citep{Geddes2014b, Kim2017}. I expect the military as the collegial group would have converging interests and capabilities over social policies. Secondly, when an authoritarian regime has party institutions, the party institution can serve interactive functions rather than one-way. It implies that party institution is not only the platform to mobilize the support from the masses \citep{Gandhi2006}, but also the mechanisms making the party elite seek particular interests. Through the party institution, the party elite can shape his interests, which are different from the interests of other classes.

% When an authoritarian regime successfully co-opts the class that matters to its political survival, it does not need to be on track of democratization. 

\subsection{A Coalitional Theory of Welfare Provision in Autocracies}

In a democracy, all constituent of classes has a right to vote, which a leftist party should be concerned to win the election. Also, citizens have freedom of association, which can make them aggregate class interests toward extensive welfare and form labor unions. The interaction of organized power resources (union) and leftist parties will lead to higher levels of social policies in terms of universalism. However, the theory does not hold in an authoritarian regime because autocrat does not guarantee the mechanism, which citizens can transform their political resources into social pressures. In an authoritarian regime, citizens rarely have the right to vote and have freedom of association. When the power resource theory explains the welfare provisions with the demand side, supply-side explanations are required to explain why autocrat provides welfare programs to his citizens. In other words, I emphasize the response of elites as decisive for illustrating authoritarian welfare states.

Previous literature has focused on the types of authoritarian regimes to explain their various performances \citep{Charron2011a,Roller2013,Cassani2017a,Yan2019}. However, existing data sets of discrete authoritarian regime types are not free from the validity issues \citep{Wilson2014}. \citet[52]{Roller2013} shows whether and how the choice of data sets impacts the empirical results on the determinants and consequences of authoritarian regimes. Also, most of the studies to examine performances of authoritarian regimes utilizes the method of residues of which we make all unidentified regime types as one type. It makes it challenging to state which authoritarian type is different from what.

A coalition theory I develop in this paper argues that different classes that autocrats need to co-opt have different influence on welfare provisions in authoritarian regimes. Although many pieces of literature on democratization emphasizes the threats of the middle class, I focus on the working class in terms of the preference for extensive welfare. The middle class does not only tend to resist the extension of the suffrage or support jingoistic foreign policies historically \citep{Lipset1959}, but also cooperate with existing elites as becoming state officials, regional civil servants through state careers \citep{Rosenfeld2017}. On the contrary, the working class is more sensitive to the change in the distribution of wealth and provisions of welfare than the middle class since it has the members as employees who depend primarily on the labor force \citep{Korpi2006}.

After the period of the Russian Revolution, I expect autocrats should consider the working class as a necessary component of their ruling coalition to stay in power even rather than the middle class that is usually considered as an essential group for welfare extension and regime transition. The Russian Revolution was the first case that showed the possibility of the working class to be an effective group to overthrow existing elites. Thus, it is not difficult to anticipate that autocrats would consider co-opting the working class, especially when the working class is the important social group in the regime. After the Russian Revolution, the power of the working class might increase, and the more efforts of autocrats to buy their support should be required. Thus, I propose the first hypothesis:\par

\begin{hyp}[H\ref{hyp:first}] \label{hyp:first}
	If all other conditions are the same, the working class's influence on universal welfare is greater than the middle class.
\end{hyp}
%\noindent HYPOTHESIS 1: If all other conditions are the same, the working class's influence on universal welfare is greater than the middle class over time.\\ \par

%The second hypothesis attempts to reveal how the delivery of welfare varies depending on the class coalition. From \citet{Briggs1961} to \citet{Gosta1990},  subsequent studies on welfare state have focused on the functions of decommodification and stratification. The constructs of decommodification were set as old-age pensions, sickness, and unemployment programs. Studies of welfare considered these programs as social policies, which provide direct cash benefits and subsidies \citep{VanVoorhis2002}. However, the welfare state is not only about cash benefits, but also about the actual delivery of services to foster recipients' welfare. When authoritarian politics is rarely a zero-sum game \citep{Olson1993a}, delivering public services could be a more efficient strategy for political survival than the selective distribution of private goods. It is more likely to be when autocrats should buy the supports of relatively large constituencies (\citealp{BuenodeMesquita2003}, \citealp[352]{Cassani2017a}).\par

%\begin{hyp}[H\ref{hyp:second}] \label{hyp:second}
%	If all other conditions are the same, the working class's influence on cash-benefit welfare programs (e.g., old-age pensions, sickness, and unemployment) decreases compared to other classes.
%\end{hyp}
%\noindent HYPOTHESIS 2: If all other conditions are the same, the working class's influence on cash-benefit welfare programs (e.g., old-age pensions, sickness, and unemployment) decreases over time compared to other classes.\\ \par

%The way to deliver welfare to citizens can explain the differences within or between the arrangements of welfare states \citep{Castles2009}. From the Russian Revolution, the numeric power and the power resources of the working class have increased. It implies that I can expect the effect of the working class on types of welfare delivery would vary over time. As the working class increases and becomes dominant, autocrats are more likely to provide benefits through a non-cash benefit welfare program, which can be considered as `public goods' \citep{Bambra2005}. In other words, authoritarian regimes that should care larger selectorate for the potential class coalition would prefer non-cash benefits welfare programs since it is costly to afford.

%In other words, the party-based authoritarian regime will face a larger selectorate to co-opt than other types of authoritarian regimes. 

%In particular, I expect the working class would be pressure to expand welfare provisions.

%The two general hypotheses aim to show how the mechanisms of class coalition and welfare provision work in an authoritarian context. If party-based regime (institutionalized parties) provide more welfare since it targets a broader class coalition of the society, other types of the authoritarian regime would show the different variation of the welfare targeting their specific class coalition. It is expected that the incentives of welfare provision might make targetable sources of welfare more valuable. The two hypotheses can also test the alternative hypothesis of cooptation in the authoritarian regime. Autocrats usually suffer from the uncertainty of longevity and succession \citep{Brownlee2007b}. Where longevity means the time horizon of incumbent autocrats, the succession problem is about the post-tenure issue of autocrats. Also, personalist authoritarian regimes are more likely to rely on personalized factions to rule and maintain power. It means that they have less institutionalized measures for cooptation for their selectorate, potential rivals, or revolts. However, personalist autocrats also need specific rents for their coalition. The autocrats would prefer other measures for private goods as it is costly for them. Thus, autocrats with the less institutionalized world may utilize the path like old-pension programs to deliver the resources to their core supporters. Providing pension and other social policy programs that cover the core groups for coalition can be an effective strategy for the autocrats' political survival \citep{Knutsen2018}. It is an alternative hypothesis, which is against the theoretical backgrounds of class coalition mechanism that the level of welfare provision increases as the size of coalition increases.\par

%In sum, I expect that party institutionalization matters as it can show how the class coalition in authoritarian regimes works. However, the argument of this paper is distinguishable from \citet{Rasmussen2019}. Although parties are the common factor, which some authoritarian regimes and democracies both have institutionalized parties, \textit{per se} cannot explain the whole world of welfare across authoritarian regimes and democracies. Like the particular indicator of the \textit{three worlds of welfare state}, party institutionalization also indicates particular and empirical aspects of the mechanism of the class coalition. Also, the class coalition mechanism can extend to not only how universal the welfare programs of the regime are, but also which programs the regime is mainly concerned with.

\subsection{Alternative Explanations}

Recent works propose a potential alternative explanation for authoritarian welfare programs. Focusing the capacity of parties to provide public goods, \citet{Bizzarro2018} argue that highly institutionalized parties can contribute to fostering economic performances. According to \citet{Bizzarro2018}, parties are not only the platforms that make leaders and members be tied strongly, and also the mechanisms to have a relatively long time horizon through solving coordination problems. This piece of research implies that institutionalized parties can absorb the influence of classes on welfare programs. Furthermore, \citet{Rasmussen2019} explains the variations of welfare universalism by focusing on party institutionalization, which means (1) how decisions in parties are based on clear, stable rules and (2) how the decisions are informed through the networks linking party elites with other broad constituencies of the party. The main argument of the article is on the same line with the class coalition and cooptation. As "institutionalized parties allow politicians to overcome coordination problems, avoid capture by special interests and form stable linkages with broad social groups, institutionalized parties enable and make political elites to pursue extensive welfare policies" even under an authoritarian regime. In other words, the more autocrats with institutionalized parties want to maintain power, the more likely they are to run universal social policies. On the contrary, if the authoritarian regime personalizes the power for the few (without or with less institutionalized parties), the more selective programs targeting the particular groups of supporters will be. 

For example, \citet{Orenstein2008} states that Communist governments provided much more social benefits such as full employment to their citizens than did other types of authoritarian regimes. Also, \citet{Morgenbesser2017} shows that the ruling party of Singapore utilizes a diverse range of cooptation measures designed to elicit greater compliance amongst citizens, opposition members, and political elites. The strategy of cooptation includes social policies that subsidize education and primary medical care, but also comprehensive public housing. Nevertheless, an authoritarian regime does not fully rely on political parties. It shows that the existence of institutionalized parties can affect the social policies of welfare. Since Major General Park Chung-hee led a coup and captured power in 1961, the South Korean government was occupied by the previous or current members of the military. Although South Korea calls this era as the period under the military regime, effective but constrained, political parties and legislature worked. Even though the level of constraints was much less than other advanced democracies, the South Korean government should consider the parties and legislature to enact laws. The military regime used some appeasement measures to control the workers and to win popular support. It shows the military regime’s embodiment of welfare state ideals \citep[90-93]{Tang2000}. For the same reason that the institutionalized party can help autocrats to distribute resources to their citizens, I establish the subsequent hypothesis:\par

\begin{hyp}[H\ref{hyp:third}] \label{hyp:third}
	If all other conditions are the same, the higher levels of party institutionalization increase the universality of welfare programs.
\end{hyp}
%\noindent HYPOTHESIS 3: If all other conditions are the same, the higher levels of party institutionalization increases the universality of welfare programs.\\ \par
%\begin{hyp}[H\ref{hyp:fourth}] \label{hyp:fourth}
%	If all other conditions are the same, the working class's influence on universal welfare is greater than the middle class when party institutionalization increases.
%\end{hyp}
%\noindent HYPOTHESIS 4: If all other conditions are the same, the working class's influence on universal welfare is greater than the middle class over time when party institutionalization increases.\\ \par


\section{Data and Empirical Specification}
\subsection{Sample Selection}
\subsubsection{Authoritarian Regimes}

I utilize the lexical indices of the Skaaning's in the V-dem data set (\textit{e\_lexical\_index}). The variable is a combined index of six binary indicators. Thus, only the scale of 6 shows universal suffrage with a competitive, party-free, fair election country. Since it covers from 1800-2017, I expect these criteria help to obtain sufficient size of the sample. Figure \ref{fig:plot1} shows that this binary classification of political regimes provides a relatively clear distinction between the two categories along with alternative continuous regime variable, electoral democracy index of V-dem.\\\par

\begin{figure}[!htbt]
	\centering
	\includegraphics[width=0.85\linewidth]{"../3. Datasets_Codebooks/Figures/Plot1"}
	\caption{The distribution of democracies and autocracies}
	\label{fig:plot1}
\end{figure}

\subsubsection{Temporal Coverage}

Since autocrats do not rule with coercion alone, they need support. \citet{Cassani2017a} argues that autocrats attempt to invest considerable efforts and resources to buy `popular' supports. It is necessary to understand the targets that autocrats want to co-opt. The popular supports, unlike democracies, can imply a different population of society because the autocrats would only want to obtain minimally necessary supports. For the autocrats, it is inefficient to consume resources to obtain unnecessary support. Then, how and when are the different classes organized and effective for autocrats? 

\begin{figure}[!htbt]
	\centering
	\includegraphics[width=1\linewidth]{"../3. Datasets_Codebooks/Figures/Plot3"}
	\caption{Time trends of the numbers of states by classes}
	\label{fig:plot2}
\end{figure}

Figure \ref{fig:plot2} shows how the numbers of classes that are considered to be co-opted to rule in authoritarian regimes vary over time. As mentioned above, the working class before the year of 1917, the Russian Revolution, is not politically significant and effective groups for ruling the regimes.\footnote{Before the Russian Revolution (1917), the cases of the working class as important groups for the regime are Paraguay (1814-1840), Bolivia (1848-1857, 1871), Costa Rica (1882-1889), Mexico (1910-1916).} Previous literature also stresses the temporal dynamics of classes on the welfare state. \citet{Rasmussen2018} show that the working class mobilized their political power through unions in the late nineteenth century. Utilizing the cases of Ghent systems of unemployment insurance, they assert that the historical records of Western European states suggests that timing matters: the level of power resource institutionalization varies by periods \citep[821-822]{Rasmussen2018}. Because I expect the working class would build a coalition to foster welfare programs, it is reasonable to restrict the sample after the Russian Revolution, which is an event showing working classes matter. 


\subsection{Welfare Programs}

A vital contribution of the analysis I develop is to clarify why authoritarian regimes design social policies to promote the welfare of citizens. I begin with exploring welfare programs in terms of the \textit{Universalism}. The theoretical concept of the universalism of welfare can be traced back to \citet{Titmuss1974}. He approaches welfare programs qualitatively and normatively. He argues that social policy should be analyzed by its necessity, culture, tradition, and history. Universalism states society should carry the crucial responsibility of the failure of the individual citizens, and it is regarded as a sort of altruism with which everyone can be beneficiary of such needs. 

\citet[25]{Gosta1990} tries to clarify the concept referring ``all citizens are endowed with similar rights, irrespective of class or market position.'' and describes that socialist regime as universalism since it ``exhibit[s] the lowest level of benefit differentials'' \citep[69]{Gosta1990}. Others manipulate the universalism from a different angle, rephrasing it as welfare generosity. They define welfare state generosity as the ratio of social transfers to government expense over the ratio of the non-working to the total population \citep{Iversen2000,Rueda2008,Yi2014}. However, the concept of welfare generosity is only able to capture the income-based effects of welfare, and it is only available for democracies. In this analysis, following \citet{Knutsen2018} and \citet{Rasmussen2019}, I measure universalism of authoritarian welfare state using alternative measurements to test the extended theoretical arguments.

First, I utilize the Social Policy around the World (SPaW) Database, which is developed by \citet{Rasmussen2016}. The \textit{Universal Index} as formal rules in an authoritarian regime, meaning the extent to which groups in society would be included in the welfare programs and the extent to the degree to which all citizens are eligible for a benefit independent of their labor market status. The \textit{Universal Index,} of SPaW, asks ``the degree to which all citizens are eligible for a benefit independent of their labor market status.'' It is expected to capture which parts of the population are included on equal terms in the same program. For example, contribution-based or employment-based programs covering one major group are scored 2. Scores of 3 indicate that programs are contribution-based or employment-based and cover two major groups, and so on, up to 8- scores where seven groups are covered. Maximum 9-scores indicate all residents are automatically entitled to benefits, a fully universal system \citep{Rasmussen2016,Rasmussen2019}. Thus, the U.I. of SPaW shows the extent of universal coverage in welfare programs.

Secondly, I also use an alternative variable to measure the level of welfare universalism from the V-dem data set. The U.I. of V-dem relies on expert coding that asks ``how many welfare programs are means-tested and how many benefits all (or virtually all) members of the polity.'' The value of 0 means there are no, or extremely limited, welfare state policies. The maximum value of 5 states that almost all welfare state policies are universal. It is measured as ordinal and converted to the interval. Although this \textit{Universal Programs} cannot tell the variations of different welfare programs, it is relatively advantageous to test the first hypothesis (H\ref{hyp:first}) compared to other measurements such as a total sum of social expenditure. The U.I. of V-dem shows the compositions of welfare programs  that a state has.

%The dataset covers 154 countries from 1790 to 2013. \citep{Rasmussen2016}. I would utilize the \textit{Encompassingness} and \textit{Universalism} indices as both aggregated form and separate form by programs. I expect this would show the relative levels of welfare \textit{Encompassingness} and \textit{Universalism} across authoritarian regime-year units.\par

\subsection{Class coalition}

This paper focuses on the different influences of class coalitions on welfare programs since classes are expected to make social pressure on decision-makers based on their distinctive interests. It is assumed that individuals who belong to similar socioeconomic strata are more likely to have converging preferences over social policies \citet[1495]{Dahlum2019}, capturing which class is the critical supporters for autocrats to maintain in power. Following the previous relevant literature, I identify the classes that are expected to affect the levels of the authoritarian welfare state into four: \textit{Working classes, Urban middle classes, Party elites, and the Military}.

I use the variable measuring regime's most important support group (\textit{v3regimpgroup}) of the V-dem data set. The variable is measured by asking, ``which (one) group does the current political regime rely on most strongly in order to maintain power.'' It contains categories of which, if it were to retract its support to the regime, would most endanger the regime. Although \textit{Class coalition} shows various classes that matter for the political survival of regimes, I manipulate the variables \textit{v3regimpgroup} according to theoretical expectations. I combined the agrarian elites, including rich peasants and large landholders, with the local elites, including chiefs. Also, the urban working classes, including labor unions and rural working classes, including peasants, are merged into one class called \textit{Working classes}.

Additionally, I define other three groups as \textit{party elites} of the party or parties that control the executive, \textit{the military}, and \textit{the urban middle classes}. As I expect the interests between the urban classes and rural middle classes would be different \citep{Dahlum2019}, the rural middle classes are excluded.
% Appendix reports the descriptive characteristics of residual classes that are not used in the primary analysis.

\begin{figure}[!htbt]
	\centering
	\includegraphics[width=1\linewidth]{"../3. Datasets_Codebooks/Figures/Plot2"}
	\caption{The Distribution of Universalism indices by Classes in Autocracies}
	\label{fig:plot3}
\end{figure}

Figure \ref{fig:plot3} is the distribution of universalism indices of the SPaW dataset and the V-Dem dataset by classes. The \textit{Universal Index} of the left panel measures how many social groups each welfare program (old-age pension, mater, sick, unemployment, working accident, family) covers. In the left panel, the urban middle class shows the highest mean value of \textit{Universal Index}. Party elites, working classes, and the military follow. Otherwise, when we use the \textit{Universal Index} of Vdem, the right panel indicates slightly different features that the working class to provide universal welfare than the urban middle class. The party elites show a lower level of universality than in the left panel but still has a positive value, which means some of the programs are cash-transfer programs. The military has the least universal welfare programs in both. Also, we can say that the military class provides significant portions of welfare programs as cash-benefits. The two different panels show different aspects of welfare universality. The left panel shows the welfare programs of the urban class-led autocracies provide the welfare programs covering various groups of societies than other classes. However, the way of delivery shows lower levels of universality than the working class. In sum, Figure \ref{fig:plot3} suggests a possibility that the working class-led autocracies provide welfare programs targeting specific social groups. At the same time, however, the programs benefit everyone who belongs to the groups.


\subsection{Party Institutionalization}

To show how autocrats co-opt the potential class coalition in selectorate, I include \textit{party institutionalization index} (P.I.) that comes from the V-Dem data set. Political parties are the platforms to connect the elites and the masses. The P.I. is reviewed in \citet{Bizzarro2018} and \citet{Rasmussen2019}. P.I. consists of several indicators, which capture the various features\footnote{The P.I. is the upper-level measurement, which is combined with five lower-level indicators of V-dem. The P.I. is "formed by adding the indicators for party organizations (\textit{v2psorgs}), party branches (\textit{v2psprbrch}), party linkages (\textit{v2psprlnks}), distinct party platforms (\textit{v2psplats}), and legislative party cohesion (\textit{v2pscohesv}), after standardization." \citep[2]{Bizzarro2017}} of the main parties in a political system.\par

P.I. enlarges the comparability of party institutionalization for the broadest available set of countries and years from 1900 to the present from 170 countries \citep[9]{Rasmussen2019}. Therefore, this data set is more appropriate for extensive cross-sectional studies, especially for developing countries. In terms of welfare provisions, the institutionalized party can absorb the influence of class coalition on welfare programs since institutionalized parties can be used as a mean of cooptation.\par

%By theoretical argument of authoritarian cooptation, the institutionalized parties are meaningful only when it is understood in the context of the supply dimension. The elites of authoritarian regimes provide the institutionalized parties to the constituents unilaterally. When I utilize the P.I, as a combined one, it can lead to biased results since the working mechanisms behind the parties might be different between democracies and authoritarian regimes. Thus, I only take the Distinct platforms (\textit{v2psplats}) concerns how many parties among those with representation in the legislature have publicly available, and distinct, party platforms. Also, I include the Constituency linkages (\textit{v2psprlnks}), which considers the most common form of linkage between parties and their constituents across all major parties.

%I exclude the other three indicators of P.I. as it is not appropriate to explain the research question of this project. For example, the Party organization, which considers how many parties have permanent organizations is less critical as "parties existing outside the legislature are ignored because these parties are not instruments of co-optation" in authoritarian regimes \citep[1285]{Gandhi2007}.\par

\subsection{Control Variables and Benchmark Model}

Alongside the critical predictors of interest, I include several variables for which I think there is a particularly strong theoretical rationale for considering as confounders, which are economic and demographic variables traditionally used in the welfare state.  First, I control for the GDP (\textit{lnGDP}), defined as the log of Gross Domestic Product per capita (in constant U.S. dollars, base 2011) to account for Wagner’s Law. The \textit{lnGDP} is taken from the Maddison Project (\citealp{Bolt2014}; \citealp[14]{Rasmussen2019}). Secondly, I include the variables of resource dependency and the log of population (\textit{lnPop}) from \citet{Miller2015a} to control the effect of unearned income and the size of population on welfare programs.

%First, many studies account for the influence of globalization on the development of welfare state \citep{Lee1999, Burgoon2001a, Rudra2005, Yi2014}. Globalization in this article is operationalized by the level of trade integration: Here, the measure of trade openness is the sum of the total imports and exports as a share of a country’s GDP.\par

\par


%include three demographic control variables, the percentage of Elderly Population (65 years old or older), the percentage of the Youth Population (under 15 years of age) because health care, social security, and education spending are sensitive to how many people are old or young. 

%The military can be served as an alternative institution to rule. However, it is difficult to measure how much the military matters in the decision-making process in a regime. I include the size of the military to control the influence of the military on welfare programs. Also, I include resource dependence from oil, diamonds, and agriculture to control the effect of `unearned' income on welfare programs. Some argue that an authoritarian regime with a high level of revenues from natural resources may show more extensive welfare programs or less repression \citep{Ross2006, Wright2008, LaPorte2017}. The size of the military, resource dependence come from \citet{Miller2015a}. The data for all the control variables mentioned above, except for the Maddison Project and \citet{Miller2015a}, comes from The World Bank World Development Indicator. 

I test the hypotheses with an unbalanced, pooled time-series cross-sectional (TSCS) data that cover 95 autocracies, respectively, during the period 1917–2000. TSCS data has the advantage of being able to track the correlation of various variables simultaneously with the differences across units and the longitudinal changes of individual units. However, TSCS data with OLS regression can lead to several problems such as heteroskedasticity and autocorrelation, which produce inefficient estimates.\footnote{Prior to empirical analysis, I test the possible problems of panel-level heteroscedasticity with likelihood ratio tests and of autocorrelation with Wooldridge test. The two tests suggest that both panel-level heteroscedasticity and autocorrelation issues can exist.} First, I estimate the models with panel-corrected standard errors to deal with heteroscedasticity problem. \citet{Beck1995} introduces an econometric technique that uses ordinary least squares (OLS) regression with the lagged dependent variable, unit, and period dummies and calculates panel-corrected standard errors. Next, I include unit dummies (here, state dummies) to control unit-specific varying factors that are difficult to observe and quantify. Also, I manipulate three-period dummies to control exogenous global evens that are expected to affect welfare programs such as World War I, II, and the Cold war period.

Possible autocorrelation is another problem to solve. \citet{Beck1995} suggest to include a lagged dependent variable. However, when models take the lagged dependent variable, the models would estimate the change of the dependent variable, not the levels of the dependent variable. Since this study attempts to trace how the levels of welfare programs in terms of universality vary, I do not include the lagged dependent variable. Instead, following the suggestions of \citet{Plumper2005}, I use the Prais–Winsten transformation to deal with the autocorrelation problem, assuming the first-order autocorrelation within panels. Lastly, all explanatory variables are lagged by three years to control for the potential exogenous effects of welfare programs.

\section{Empirical Analysis}
Table \ref{tab:table1} shows the results of testing the first and the third research hypotheses. Model 1 on the \textit{Universal Index} of \textit{SPaW} data set shows that all classes cover more social groups within regimes compared to the middle class. The working class covers a broader range of social groups to cover with welfare programs than the urban middle class. Model 2 tests the second hypothesis of party institutionalization. When I use \textit{Universal Index} of \textit{SPaW}, party institutionalization does not have statistical significance. 

\citet[16]{Rasmussen2019} indicate that party institutionalization can foster universal welfare programs. However, when I test with authoritarian regimes, it is difficult to argue that institutionalized party increases universal welfare programs. Model 3 tests the competing two hypotheses of Model 1 and Model 2 with the dependent variable of \textit{SPaW}. When I control the party institutionalization, the working class and the military have no statistical significance, which implies they are not distinctive from the urban middle class. Otherwise, party elites are still significant that they show broader universal coverage of welfare programs than the urban middle class.

\begin{table}[htbp]\centering
	\def\sym#1{\ifmmode^{#1}\else\(^{#1}\)\fi}
	\caption{Table 1: Class Coalitions, Party Institutionalization and Welfare Universalism}
	\resizebox{\textwidth}{!}{\begin{tabular}{l*{6}{c}}
			\hline\hline
			&\multicolumn{1}{c}{(1)}&\multicolumn{1}{c}{(2)}&\multicolumn{1}{c}{(3)}&\multicolumn{1}{c}{(4)}&\multicolumn{1}{c}{(5)}&\multicolumn{1}{c}{(6)}\\
			&\multicolumn{1}{c}{Universalism Index of SPaW}&\multicolumn{1}{c}{Universalism Index of SPaW}&\multicolumn{1}{c}{Universalism Index of SPaW}&\multicolumn{1}{c}{Universalism Index of V-Dem}&\multicolumn{1}{c}{Universalism Index of V-Dem}&\multicolumn{1}{c}{Universalism Index of V-Dem}\\
			\hline
			\textit{Reference: Urban middle}             &                     &                     &                     &                     &                     &                     \\
			&                     &                     &                     &                     &                     &                     \\
			[1em]
			Party elites        &       0.934\sym{***}&                     &       0.738\sym{***}&       0.034         &                     &       0.013         \\
			&     (0.207)         &                     &     (0.222)         &     (0.026)         &                     &     (0.028)         \\
			[1em]
			The military        &       0.391\sym{*}  &                     &      -0.077         &       0.014         &                     &      -0.007         \\
			&     (0.196)         &                     &     (0.216)         &     (0.023)         &                     &     (0.029)         \\
			[1em]
			Urban working       &       1.356\sym{**} &                     &       1.353\sym{**} &      -0.038         &                     &      -0.066         \\
			&     (0.422)         &                     &     (0.481)         &     (0.074)         &                     &     (0.080)         \\
			[1em]
			Party Institutionalization&                     &       1.411         &       1.435         &                     &       0.717\sym{***}&       0.693\sym{***}\\
			&                     &     (0.869)         &     (0.788)         &                     &     (0.134)         &     (0.135)         \\
			[1em]
			Ln.Pop.             &       1.121\sym{***}&       0.897\sym{**} &       0.885\sym{**} &       0.236\sym{***}&       0.248\sym{***}&       0.254\sym{***}\\
			&     (0.291)         &     (0.339)         &     (0.308)         &     (0.033)         &     (0.036)         &     (0.037)         \\
			[1em]
			Ln.GDPpc            &       8.071\sym{***}&       8.806\sym{***}&       8.610\sym{***}&       0.720\sym{***}&       0.712\sym{***}&       0.708\sym{***}\\
			&     (0.569)         &     (0.710)         &     (0.695)         &     (0.050)         &     (0.060)         &     (0.060)         \\
			[1em]
			Resource Dep.       &      -0.015\sym{*}  &      -0.019\sym{**} &      -0.019\sym{**} &       0.000         &       0.000         &       0.000         \\
			&     (0.006)         &     (0.006)         &     (0.006)         &     (0.001)         &     (0.001)         &     (0.001)         \\
			[1em]
			Country Dummies     &         Yes         &         Yes         &         Yes         &         Yes         &         Yes         &         Yes         \\
			[1em]
			WWI Dummies         &         Yes         &         Yes         &         Yes         &         Yes         &         Yes         &         Yes         \\
			[1em]
			WWII Dummies        &         Yes         &         Yes         &         Yes         &         Yes         &         Yes         &         Yes         \\
			[1em]
			Cold War Dummies    &         Yes         &         Yes         &         Yes         &         Yes         &         Yes         &         Yes         \\
			\hline
			Observations        &        2586         &        2187         &        2187         &        4175         &        3562         &        3562         \\
			\(R^{2}\)           &       0.927         &       0.930         &       0.942         &       0.649         &       0.704         &       0.693         \\
			\hline\hline
			\multicolumn{7}{l}{\footnotesize Panel-Corrected Standard Errors reported in parentheses.}\\
			\multicolumn{7}{l}{\footnotesize OLS with panel corrected standard errors. Constant, period dummies and country dummies not displayed.}\\
			\multicolumn{7}{l}{\footnotesize \sym{*} \(p<0.05\), \sym{**} \(p<0.01\), \sym{***} \(p<0.001\)}\\
	\end{tabular}}
	\label{tab:table1}
\end{table}

From Model 4 to Model 6, I utilize the \textit{Universal Index} of \textit{V-Dem} as dependent variable. By the operationalized definition, the variable tells how much the delivery of welfare programs lean toward cash-benefits or universalism. If the \textit{Universal Index} of \textit{V-Dem} is greater than 0, then it means the regime has more universal welfare programs for everyone than cash-benefit programs targeting a particular group of people. In Model 4, all classes are not statistically significant compared to the urban middle class. It suggests that the ways to deliver welfare across class coalitions are not distinctive to each other.

As the P.I. show statistically significant regardless of the data sources in \citep{Rasmussen2019}, I disaggregate the P.I. into several indicators to explore these dispersed results. Institutionalized parties under authoritarian rules may not be identical to parties under democracies. Figure \ref{fig:plot4} shows the association between disaggregated P.I. and \textit{Universal Index} of \textit{V-Dem} under authoritarian regimes when I control class coalitions and other covariates. Unlike the previous work of \citet{Rasmussen2019}, three indicators of P.I. show statistical insignificance---\textit{linkage, organizations, and platforms}. The \textit{branch and cohension} indicators are significant. The last panel of Figure \ref{fig:plot4} means that authoritarian regimes deliver welfare in the universal way when they have more permanent local branches and more cohesive memberships. 

The different findings of Table \ref{tab:table1} do not confirm both hypotheses when I concern the influence of alternative explanation. When autocrats make coalitions with the working class, they provide welfare targeting broader social groups than the urban middle class coalitions. However, the institutionalized party, which is considered as an essential mechanism for resource distribution in existing studies, can affect the relationship between class coalitions and universal welfare programs. However, not all aspects of the institutionalized party enhance universal welfare programs. When the parties have more permanent local branches or have more cohesive memberships, authoritarian regimes are more likely to have welfare programs that benefit everyone.

\begin{figure}[!htbt]
	\centering
	\includegraphics[width=0.85\linewidth]{"../3. Datasets_Codebooks/Figures/Plot4"}
	\caption{Disaggregated Party Institutionalization and Welfare Universalism (95\% C.I.)}
	\label{fig:plot4}
\end{figure}

To understand the dynamics between the effects of working class coalitions and institutionalized parties in authoritarian regime, I estimate the interaction between working class and party institutionalization. Figure \ref{fig:plot5} indicates the effect of working class coalitions are mediated by the levels of party institutionalization. As parties are more institutionalized, an authoritarian regime which builds coalition with working class provides universal welfare when it has more institutionalized parties. 

\begin{figure}[!htbt]
	\centering
	\includegraphics[width=0.85\linewidth]{"../3. Datasets_Codebooks/Figures/Plot5"}
	\caption{Predicted Universal Welfare of Working Class by Party Institutionalization}
	\label{fig:plot5}
\end{figure}

%\small
\setlength\tabcolsep{2pt}
\def\sym#1{\ifmmode^{#1}\else\(^{#1}\)\fi}
\afterpage{
	\begin{longtable}in{longtable}}[!htbt]{l*{2}{c}}
	\centering
	\caption{Class Coalitions, Disaggregated Party Institutionalization and Welfare Universalism} \label{tab:table2}\\
			\toprule
			&\multicolumn{1}{c}{\bf \textit{SPaW}} & \multicolumn{1}{c}{\bf \textit{V-Dem}}  \\
			&\multicolumn{1}{c}{Model 1}           & \multicolumn{1}{c}{Model 2}         \\
			\endfirsthead
			\midrule
			&\multicolumn{1}{c}{\bf \textit{SPaW}} & \multicolumn{1}{c}{\bf \textit{V-Dem}}  \\
			&\multicolumn{1}{c}{Model 1}           & \multicolumn{1}{c}{Model 2}         \\
			\endhead
			\hline \multicolumn{3}{l}{\textit{Continued on next page}} \\
			\endfoot
			\hline \hline
			\endlastfoot
			Party elites        &       0.911\sym{***}&       0.039         \\
			                    &     (0.214)         &     (0.027)         \\
			[1em]
			The military        &       0.234         &       0.025         \\
			                    &     (0.207)         &     (0.025)         \\
			[1em]
			Working class       &       0.767\sym{*}  &      -0.038         \\
			                    &     (0.371)         &     (0.051)         \\
			[1em]
			Party Branch        &       0.215         &       0.053\sym{**} \\
			                    &     (0.134)         &     (0.019)         \\
			[1em]
			Linkage             &      -0.054         &       0.011         \\
			                    &     (0.137)         &     (0.013)         \\
			[1em]
			Organization        &      -0.067         &      -0.008         \\
			                    &     (0.139)         &     (0.019)         \\
			[1em]
			Platform            &      -0.394\sym{**} &       0.007         \\
			                    &     (0.123)         &     (0.019)         \\
			[1em]
			Cohension           &       0.727\sym{***}&       0.070\sym{***}\\
			                    &     (0.135)         &     (0.018)         \\
			[1em]
			Control Variables   &         Yes         &         Yes         \\
			[1em]
			Country Dummies     &         Yes         &         Yes         \\
			[1em]
			WWI Dummies         &         Yes         &         Yes         \\
			[1em]
			WWII Dummies        &         Yes         &         Yes         \\
			[1em]
			Cold War Dummies    &         Yes         &         Yes         \\
			\hline
			Observations        &        2571         &        4130         \\
			\(R^{2}\)           &       0.918         &       0.625         \\
			\hline
			\multicolumn{3}{l}{\footnotesize Standard errors in parentheses}\\
			\multicolumn{3}{l}{\footnotesize OLS with panel corrected standard errors.}\\ 
			\multicolumn{3}{l}{\footnotesize Panel-Corrected Standard Errors reported in parentheses.}\\
			\multicolumn{3}{l}{\footnotesize Constant, period dummies and country dummies not displayed.}\\
			\multicolumn{3}{l}{\footnotesize \sym{*} \(p<0.05\), \sym{**} \(p<0.01\), \sym{***} \(p<0.001\)}\\
			\bottomrule
	\end{longtable}

%The second hypothesis states that the working class's influence on cash-benefit welfare programs (e.g., old-age pensions, sickness, and unemployment) is the least compared to other classes. This hypothesis is built on the previous welfare-state literature expecting the working class is sensitive to the labor environment. Existing studies argue that the working class does not only support universal welfare programs, but also prefer long-lasting benefits which can improve their conditions. Thus, I test the second hypothesis with six-disaggregated welfare programs (\textit{old-age pension, mater, sickness, working incidences, unemployment, family}) as dependent variables. If autocrats prefer to provide private goods when they expect smaller winning coalition and try to provide public goods vice versa, 

%According to the hypothesis, the working class has greater influence 

%\documentclass[11pt]{article}
\usepackage{setspace}
\doublespacing
\usepackage{geometry}
\usepackage{titling}
\usepackage{blindtext}
\geometry{margin=1in}
\usepackage{graphics} % for pdf, bitmapped graphics files
\usepackage{epsfig} % for postscript graphics files
\usepackage{mathptmx} % assumes new font selection scheme installed
\usepackage{times} % assumes new font selection scheme installed
\usepackage[fleqn]{amsmath} % assumes amsmath package installed
\usepackage{amssymb}  % assumes amsmath package installed
\usepackage[affil-it]{authblk}
\usepackage{natbib}
\bibpunct{(}{)}{;}{a}{}{,}
\usepackage{hyperref}
\usepackage{bookmark}
\hypersetup{
	colorlinks   = true, %Colours links instead of ugly boxes
	urlcolor     = blue, %Colour for external hyperlinks
	linkcolor    = blue, %Colour of internal links
	citecolor   = blue %Colour of citations,
}
\usepackage{xcolor}
\usepackage{lettrine}
\usepackage{longtable}
\usepackage{afterpage}
\usepackage{booktabs}
\usepackage[utf8]{inputenc}
\usepackage[sc]{mathpazo}
\usepackage[T1]{fontenc}
\usepackage{adjustbox}
\usepackage{ntheorem}
\theoremseparator{:}
\newtheorem{hyp}{Hypothesis}
\usepackage[toc,title,page]{appendix}

\makeatletter % <=======================================================
\renewcommand\@seccntformat[1]{}
\renewcommand{\@makefntext}[1]{%
	\setlength{\parindent}{0pt}%
	\begin{list}{}{\setlength{\labelwidth}{6mm}% 1.5em <==================
			\setlength{\leftmargin}{\labelwidth}%
			\setlength{\labelsep}{3pt}%
			\setlength{\itemsep}{0pt}%
			\setlength{\parsep}{0pt}%
			\setlength{\topsep}{0pt}%
			\footnotesize}%
		\item[\@thefnmark\hfil]#1% @makefnmark
	\end{list}%
}

\makeatother % <========================================================


\begin{document}
	
\begin{table}[htbp]\centering
\def\sym#1{\ifmmode^{#1}\else\(^{#1}\)\fi}
\caption{Class Coalitions, and Welfare Programs}
\begin{tabular}{l*{6}{c}}
\hline\hline
                    &\multicolumn{1}{c}{(1)}         &\multicolumn{1}{c}{(2)}         &\multicolumn{1}{c}{(3)}         &\multicolumn{1}{c}{(4)}         &\multicolumn{1}{c}{(5)}         &\multicolumn{1}{c}{(6)}         \\
                    &     Old-age         &       Mater         &        Sick         &     Working         &     Unempl.         &      Family         \\
\hline
Working class       &       0.075         &       0.158         &       0.220\sym{*}  &      -0.019         &      -0.062         &       0.031         \\
                    &     (0.091)         &     (0.088)         &     (0.086)         &     (0.059)         &     (0.061)         &     (0.060)         \\
[1em]
Urban middle        &       0.043         &      -0.053         &       0.058         &      -0.110         &      -0.087         &       0.015         \\
                    &     (0.128)         &     (0.123)         &     (0.106)         &     (0.069)         &     (0.278)         &     (0.085)         \\
[1em]
Party elites        &       0.111\sym{*}  &       0.149\sym{***}&       0.145\sym{**} &       0.101\sym{***}&       0.060         &       0.056         \\
                    &     (0.045)         &     (0.037)         &     (0.046)         &     (0.029)         &     (0.057)         &     (0.031)         \\
[1em]
Ln.Pop.             &       0.270\sym{***}&       0.064         &       0.124         &       0.014         &       0.126\sym{**} &       0.070         \\
                    &     (0.073)         &     (0.066)         &     (0.070)         &     (0.043)         &     (0.043)         &     (0.044)         \\
[1em]
Resource Dep.       &      -0.002         &      -0.002         &       0.001         &      -0.001         &      -0.002         &      -0.001         \\
                    &     (0.002)         &     (0.002)         &     (0.002)         &     (0.001)         &     (0.001)         &     (0.001)         \\
[1em]
Country Dummies     &         Yes         &         Yes         &         Yes         &         Yes         &         Yes         &         Yes         \\
[1em]
WWI Dummies         &         Yes         &         Yes         &         Yes         &         Yes         &         Yes         &         Yes         \\
[1em]
WWII Dummies        &         Yes         &         Yes         &         Yes         &         Yes         &         Yes         &         Yes         \\
[1em]
Cold War Dummies    &         Yes         &         Yes         &         Yes         &         Yes         &         Yes         &         Yes         \\
[1em]
P.I. Indicators     &         Yes         &         Yes         &         Yes         &         Yes         &         Yes         &         Yes         \\
\hline
Observations        &        3381         &        3309         &        3313         &        3037         &        3421         &        3406         \\
\(R^{2}\)           &       0.745         &       0.775         &       0.733         &       0.906         &       0.884         &       0.850         \\
\hline\hline
\multicolumn{7}{l}{\footnotesize Standard errors in parentheses}\\
\multicolumn{7}{l}{\footnotesize OLS with panel corrected standard errors. Panel-Corrected Standard Errors reported in parentheses.}\\
\multicolumn{7}{l}{\footnotesize Constant, period dummies and country dummies not displayed.}\\
\multicolumn{7}{l}{\footnotesize \sym{*} \(p<0.05\), \sym{**} \(p<0.01\), \sym{***} \(p<0.001\)}\\
\end{tabular}
\end{table}
\end{document}

%By theoretical argument of authoritarian cooptation, the institutionalized parties are meaningful only when it is understood in the context of the supply dimension. The elites of authoritarian regimes provide the institutionalized parties to the constituents unilaterally. When I utilize the P.I, as a combined one, it can lead to biased results since the working mechanisms behind the parties might be different between democracies and authoritarian regimes. Thus, I only take the Distinct platforms (\textit{v2psplats}) concerns how many parties among those with representation in the legislature have publicly available, and distinct, party platforms. Also, I include the Constituency linkages (\textit{v2psprlnks}), which considers the most common form of linkage between parties and their constituents across all major parties.

%Table \ref{tab:table2} takes similar model specification with different dependent variables, disaggregated welfare programs. According to the theoretical framework of this project, a class coalition affects the paths of welfare delivery. Unlike previous research such as \citet{Rasmussen2019}, party institutionalization may not fully explain the variations of welfare programs. An institutionalized party may also be one of the outcomes, which is produced by a coalition among different classes in a society. If so, the institutionalized parties would be associated with different welfare programs in divergent ways.\par

\section{Conclusion}

This study argues that how universal welfare programs depend on the groups with which autocrats want to make a coalition. The first hypothesis states that the working class drives welfare programs compared to the urban middle class considered as a prominent class affecting the size of the welfare state. Table \ref{tab:table1} shows that the first hypothesis is partially confirmed in terms of welfare coverage meaning how many social groups get benefits from the welfare programs. However, the relationship between class coalitions without party elites and welfare universality disappears when I control the party institutionalization. Autocrats consider who to co-opt because it is possible to stay in office through successfully embracing the essential class, but it may matter whether they do so through a party institution.

Class coalitions matter when each class recognizes its distinctive interests and has incentives to achieve the interests. If so, the class, which autocrats should co-opt, can affect the decision-making of autocrats for resource allocation. The working class, which is relatively poor and large in society, would require autocrats to provide universal welfare programs to foster its condition (Model 1). 

However, Figure \ref{fig:plot5} shows that autocrats can mitigate the threats from classes through specific institutions---\textit{political parties}. Some studies explore the influence of authoritarian legislatures \citep{Jensen2014, Truex2012} and suggest that autocrats may establish such institutions for controlling and managing potential or possible oppositions in the elites through the institutional way. According to \citet{Magaloni2006}, however, institutionalized parties are also utilized for similar purposes. In authoritarian regimes, political parties are not only the machine to mobilize the support of the masses. Political parties can control the flow of information and identify the potential rivals. Hence, when political parties are the platforms that are more closely related to the masses than the legislature in authoritarian regimes, the institutional functions of parties can be applied to the masses. Thus, the influece of the working class, relative to the middle class, may depend on party strength.

Figure \ref{fig:plot4} shows that not all aspects of institutionalized parties contribute to improving universal welfare programs. They do if they have permanent local branches and have cohesive memberships. It implies that autocrats can reshape the relationship between the supporters and themselves using institution---parties. \citet{Rasmussen2019} presume that institutionalized parties will lead states to establish extensive welfare states, but it is not necessary.

This study has several limitations in terms of measurements and unobservable factors of authoritarian regimes. However, most of the studies on authoritarian regimes face the issue of data availability. Under the limitations, this preliminary analysis provides several implications. First, it requires us to revisit what constraints the choices of autocrats, and how autocrats try to overcome it. Second, it should be cautious that setting a single continuum to evaluate both democracy and authoritarian regimes can overlook significant variations within it.


\newpage
	\bibliographystyle{apsr}
	\bibliography{19AuthWelfare}


\end{document}