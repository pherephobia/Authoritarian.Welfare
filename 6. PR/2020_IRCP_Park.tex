\documentclass{Bredelebeamer}
\usepackage[utf8]{inputenc}
\usepackage[T1]{fontenc}
\usepackage{scrextend}
\usepackage{booktabs}
\changefontsizes{11pt}
\usepackage{natbib}
\usepackage{adjustbox}
\usepackage{graphicx}

\bibpunct{(}{)}{;}{a}{}{,}
%%%%%%%%%%%%%%%%%%%%%%%%%%%%%%%%%%%%%%%%%%%%%%%%
\title{Why Do Authoritarian Regimes Provide Welfare Programs?}
% Title

% Optional subtitle
\author{Sanghoon Park \inst{1}}
% The \ inst {...} command displays the speaker's affiliation.
% If there are several speakers: Marcel Dupont \ inst {1}, Roger Durand \ inst {2}
% Just add another institute to the model below.
\institute[University of South Carolina]
{
  \inst{1}%
  University of South Carolina\\Department of Political Science}
\date{\today}
% Optional. The date, usually the day of the conference
\subject{World Congress Presentation (6/26/2019)}
% It's used in PDF metadata

%%%%%%%%%%%%%%%%%%%%%%%%%%%%%%%%%%%%%%%%%%%%%%%%%%%%%%%%%%%%%%%%%%%%%
\begin{document}


\begin{frame}
  \titlepage
\end{frame}



\section{Research Question}
\begin{frame}[t]{Research Question}
	\begin{itemize}
		\item Existing Studies of welfare assume fully democratic regime.
		\begin{itemize}
			\item Compensations %\citep{Boix2001a,Adsera2002} 
			\item Median Voters %\citep{Meltzer1981,Iversen2006} 
			\item Power resources %\citep{Bradley2003,Korpi2006,Lupu2011} 
		\end{itemize}%\pause
		\item How can we explain
		\begin{itemize}
			\item why authoritarian regimes provide welfare programs?
			\item the variations of authoritarian welfare states?
		\end{itemize}
	\end{itemize}	
\end{frame}

\section{Theory}
\begin{frame}{Understanding of Welfare states}
\begin{alertblock}{Definition}
	\begin{itemize}
		\item A state in which organized power is deliberately used to modify the play of market forces.
		\item Minimum income, social insurance, and universal services
	\end{itemize}
\end{alertblock}
\begin{itemize}
	\item The structure of class coalitions presses govt. toward specific types of welfare state \citep{Gosta1990}.
	\item For example, red-green alliance of Sweden $\rightarrow$ Universal welfare programs.
\end{itemize}
\end{frame}

\begin{frame}{Welfare Regimes in Autocracies}
\begin{itemize}
	\item Leaders maintain their coalitions of supporters by public and private goods \citep{BuenodeMesquita2003}.
	\item Autocracies are not same in terms of coalition buildings \citep{Gandhi2009,Levitsky2010}.
	\item Autocrats also have incentives to provide public goods to their citizens \citep{Wintrobe1998}
\end{itemize}
\vspace{0.3in}
The problem is \textbf{\textit{who is the target autocrats should care}}. \centering 
\end{frame}

\begin{frame}{Classes}
In authoritarian regimes,
	\begin{itemize}
		\item the welfare might be the same as the means to co-opt the classes from selectorate for ruling coalitions.
		\item The extent to universal welfare programs depends on which class the regime should co-opt.
	\end{itemize}
Assumption of the class: individuals in similar socioeconomic strata are more likely to have converging preferences over social policies.
	\begin{itemize}
		\item Income based: \textbf{the middle class} and \textbf{working class}
		\item Institutional based: \textbf{The party elite} and \textbf{military}
		\end{itemize}
\end{frame}

\begin{frame}{Classes}
\begin{enumerate}
	\item The middle class can be heterogeneous group depending on whether it is based on rural or urban area. \citep{Dahlum2019}.
	\begin{itemize}
		\item The urban middle class has higher leverage for autocracies as they have motivations and capability for regime transition.
	\end{itemize}
	\item The working class
	\begin{itemize}
		\item The rural working class is not free from the urbanization or industrialization.
		\item Elites have tried to co-opt both (e.g., the Russian Revolution).
	\end{itemize}
	\item The party elite and the military
	\begin{itemize}
		\item Specific institutions create a class of elites with distinct interests and incentives.
	\end{itemize}
\end{enumerate}
\end{frame}

\begin{frame}{Class Coalitions}
\begin{itemize}
	\item The different classes that autocrats need to co-opt $\rightarrow$ Variety of welfare programs.
	\begin{itemize}
		\item Previous democratization literature emphasizes the threats of the middle class.
		\item This study focus on the working class which have preference for extensive welfare.
		\begin{itemize}
			\item The working class is more sensitive to the change in the distribution of wealth and welfare since it depends primarily on the labor force.
		\end{itemize}
	\end{itemize}
	\item Hypotheses ($\text{\textit{H}}_1$): If all other conditions are the same, the working class's inference on universal welfare programs is greater than the middle class.
\end{itemize}
\end{frame}

\begin{frame}{Alternative: Party Institutionalization}
\begin{itemize}
	\item Recent works focus on capacity of parties to provide public goods.
	\begin{itemize}
		\item The more autocrats with institutionalized parties want to maintain power, the more likely they are to run universal welfare programs \citep{Rasmussen2019}.
		\item Institutionalized party can help autocrats to distribute resources to their citizens.
	\end{itemize}
	\item Hypotheses ($\text{\textit{H}}_2$): If all other conditions are the same, the higher levels of party institutionalization increase the universlity of welfare programs.
\end{itemize}
\end{frame}

\section{Data and Empirical Specification}
\begin{frame}{Sample Selection}
Authoritarian Regimes: \centering
\begin{figure}[!htbt]
	\centering
	\includegraphics[width=0.85\linewidth]{"../3. Datasets_Codebooks/Figures/Plot1"}
	\caption{The distribution of Democracies and Autocracies}
	\label{fig:plot1}
\end{figure}
\end{frame}	

\begin{frame}{Sample Selection}
After the Russian Revolution of 1917: \centering
\begin{figure}[!htbt]
	\centering
	\includegraphics[width=0.9\linewidth]{"../3. Datasets_Codebooks/Figures/Plot3"}
	\caption{The distribution of Democracies and Autocracies}
	\label{fig:plot2}
\end{figure}
\end{frame}



\begin{frame}<beamer:0>
	\bibliographystyle{apsr}
	\bibliography{19AuthWelfare.bib}
\end{frame}
\end{document}
