\documentclass{Bredelebeamer}
\usepackage[utf8]{inputenc}
\usepackage[T1]{fontenc}
\usepackage{scrextend}
\usepackage{booktabs}
\changefontsizes{11pt}
\usepackage{natbib}
\usepackage{adjustbox}
\usepackage{graphicx}
\hyphenation{thatshouldnot}
\bibpunct{(}{)}{;}{a}{}{,}
%%%%%%%%%%%%%%%%%%%%%%%%%%%%%%%%%%%%%%%%%%%%%%%%
\title{Why Do Authoritarian Regimes Provide Welfare Programs?}
% Title

% Optional subtitle
\author{Sanghoon Park \inst{1}}
% The \ inst {...} command displays the speaker's affiliation.
% If there are several speakers: Marcel Dupont \ inst {1}, Roger Durand \ inst {2}
% Just add another institute to the model below.
\institute[University of South Carolina]
{
  \inst{1}%
  University of South Carolina\\Department of Political Science}
\date{\today}
% Optional. The date, usually the day of the conference
\subject{World Congress Presentation (6/26/2019)}
% It's used in PDF metadata

%%%%%%%%%%%%%%%%%%%%%%%%%%%%%%%%%%%%%%%%%%%%%%%%%%%%%%%%%%%%%%%%%%%%%
\begin{document}


\begin{frame}
\titlepage
\end{frame}

\section{Research Question}
\begin{frame}[t]{Research Question}
	\begin{itemize}
		\item Existing studies of welfare assume fully democratic regimes.\pause
		\begin{itemize}
			\item Compensations %\citep{Boix2001a,Adsera2002} 
			\item Median voters %\citep{Meltzer1981,Iversen2006} 
			\item Power resources %\citep{Bradley2003,Korpi2006,Lupu2011} 
		\end{itemize}\pause
		\item How can we explain the following questions?
		\begin{itemize}
			\item Why do authoritarian regimes provide welfare?
			\item Are there variations of authoritarian welfare states?
		\end{itemize}
	\end{itemize}	
\end{frame}

\section{Theory}

\begin{frame}{Welfare Regimes in Autocracies}
\begin{itemize}
	\item The structure of class coalitions presses govt. toward specific types of welfare states \citep{Gosta1990}.
	\begin{itemize}
		\item For example, red-green alliance of Sweden $\rightarrow$ universal %welfare programs.
	\end{itemize}
	\item Leaders maintain coalitions of supporters by public, private goods \citep{BuenodeMesquita2003}.
\end{itemize}
\end{frame}
\begin{frame}{Welfare Regimes in Autocracies}
\begin{itemize}
	\item Autocracies have different coalitions \citep{Gandhi2009,Levitsky2010}.
	\item Autocrats also have incentives to provide public goods \citep{Olson1993a,Wintrobe1998}
\end{itemize}
\end{frame}

\begin{frame}{Welfare Regimes in Autocracies}
\centering The problem is: \pause \textbf{\textit{who is the target?}} %autocrats should care}}. \centering 
\end{frame}

\begin{frame}{Classes}
In authoritarian regimes,
	\begin{itemize}
		\item Welfare = Co-optation
		\item Extent of welfare depends on class. %the regime should co-opt.
	\end{itemize}
Assumption: individuals in similar strata have converging preferences over social policies.
	\begin{itemize}
		\item Income-based: \textbf{middle class} and \textbf{working class}
		\item Institutional-based: \textbf{party elite} and \textbf{military}
		\end{itemize}
\end{frame}

\begin{frame}{Classes}
\begin{enumerate}
	\item \textbf{Urban middle} and \textbf{urban working} classes
	\begin{itemize}
		\item Urban classes $\rightarrow$ Higher leverages \citep{Dahlum2019}.
		\item Different social life $\rightarrow$ Different welfare demands
	\end{itemize}
	\item \textbf{Party elite} and \textbf{military}
	\begin{itemize}
		\item Specific institutions create class of elites with distinct interests, incentives.
	\end{itemize}
\end{enumerate}

\end{frame}

\begin{frame}{Class Coalitions}
\begin{itemize}
	\item Different classes $\rightarrow$ Different welfare demands.
	\begin{itemize}
		\item Democratization literature emphasizes threats of middle class.
		\item Working class prefers more extensive welfare than middle.
		\begin{itemize}
			\item Working class more sensitive to change in distribution of wealth, welfare.% since it depends primarily on the labor force.
		\end{itemize}
	\end{itemize}\pause
	\item $\text{\textit{H}}_1$: Urban working class has greater influence on universal welfare programs than middle class.
\end{itemize}
\end{frame}

\begin{frame}{Party Institutionalization}
\begin{itemize}
	\item Recent works focus on capacity of parties to provide public goods.
	\begin{itemize}
		\item Autocrats with institutionalized parties $\rightarrow$ universal welfare programs \citep{Rasmussen2019}.
		\item Institutionalized party helps distribute resources.
	\end{itemize}\pause
	\item $\text{\textit{H}}_2$: Higher levels of party institutionalization increase welfare programs universality.
\end{itemize}
\end{frame}

\begin{frame}{Class Coalition \& Party Institutionalization}
	\begin{itemize}
		\item Institutionalized parties can aggregate the demands of working classes in a more organized and efficient way.
		\item Autocrats can handle the class coalition using institutionalized party.\pause
		\item $\text{\textit{H}}_3$: Influence of urban working class on universal welfare is greater when party institutionalization increases.
	\end{itemize}
\end{frame}

\section{Data and Empirical Specification}
\begin{frame}{Sample Selection}
Authoritarian Regimes: \centering 
\begin{figure}[!htbt]
	\centering
	\includegraphics[width=0.85\linewidth]{"../3. Datasets_Codebooks/Figures/Plot1"}
	\caption{The distribution of Democracies and Autocracies}
	\label{fig:plot1}
\end{figure}
\end{frame}	

\begin{frame}{Sample Selection}
After the Russian Revolution of 1917: \centering 
\begin{figure}[!htbt]
	\centering
	\includegraphics[width=0.9\linewidth]{"../3. Datasets_Codebooks/Figures/Plot3"}
	\caption{Time trends of the numbers of states by classes}
	\label{fig:plot2}
\end{figure}
\end{frame}

\begin{frame}{Dataset and methods}
\begin{itemize}
	\item Coverage: 95 authoritarian states from 1917-2000 (unbalanced)
	\item Sources of data
	\begin{itemize}
		\item Social Policies around the World (SPaW) data set
		\item Variety of Democracies (V-Dem) data set
	\end{itemize}
	\item Variable
	\begin{itemize}
		\item DV: Welfare programs \small{(social coverage)}
		\item EV: Class coalitions \small{(urban working, urban middle, party elites, and military)}
		\item CV: Logged GDP per capita (V-Dem), logged of population, and resource dependence \citep{Miller2015}
	\end{itemize}
\end{itemize}
\end{frame}

\begin{frame}{Dataset and methods}
\begin{figure}[!htbt]
	\centering
	\includegraphics[width=1\linewidth]{"../3. Datasets_Codebooks/Figures/Plot2"}
	\caption{The distribution of universal indices by classes in Autocracies}
	\label{fig:plot3}
\end{figure}
\end{frame}

\begin{frame}{Empirical Analysis}
\begin{figure}[!htbt]
	\centering
	\includegraphics[width=1\linewidth]{"PRplot1"}
	\caption{Class coalitions, Party Institutionalization, and Welfare Universalism}
	\label{fig:plot4}
\end{figure}
\end{frame}

%begin{frame}{Empirical Analysis}
%\begin{figure}[!htbt]
%	\centering
%	\includegraphics[width=1\linewidth]{"../3. Datasets_Codebooks/Figures/Plot4"}
%	\caption{Disaggregated Party Institutionalization and Welfare Universalism (95\% CI)}
%	\label{fig:plot5}
%\end{figure}
%\end{frame}

\begin{frame}{Empirical Analysis}
\begin{figure}[!htbt]
	\centering
	\includegraphics[width=1\linewidth]{"../3. Datasets_Codebooks/Figures/Plot5"}
	\caption{Predicted Universal Welfare of No-urban and urban Working Class by Party\\Institutionalization (95\% CI)}
	\label{fig:plot6}
\end{figure}
\end{frame}
		
\section{Conclusion}
\begin{frame}{Conclusion}
	\begin{itemize}
		\item Similar paths: Class coalition $\rightarrow$ Welfare provision \pause
	\end{itemize}
	\centering
	\includegraphics[width=1\linewidth]{"P/PR 1"}
\end{frame}

\begin{frame}{Conclusion}
	\begin{itemize}
		\item Similar paths: Class coalition $\rightarrow$ Welfare provision
	\end{itemize}
	\centering
	\includegraphics[width=1\linewidth]{"P/PR 2"}
\end{frame}

\begin{frame}{Conclusion}
	\begin{itemize}
	\item However, different mechanisms: Co-optation $\rightarrow$ Welfare \pause
	\end{itemize}
	\centering
	\includegraphics[width=1\linewidth]{"P/PR 3"}
\end{frame}

\begin{frame}{Conclusion}
	\begin{itemize}
	\item However, the underlying mechanisms are different (participation/cooptation).
	\end{itemize}
	\centering
	\includegraphics[width=1\linewidth]{"P/PR 4"}
\end{frame}

\begin{frame}{Conclusion}
	\begin{itemize}
	\item However, the underlying mechanisms are different (participation/cooptation).
	\end{itemize}
	\centering
	\includegraphics[width=1\linewidth]{"P/PR 5"}
\end{frame}

\begin{frame}{Conclusion}
	\begin{itemize}
	\item However, the underlying mechanisms are different (participation/cooptation).
	\end{itemize}
	\centering
	\includegraphics[width=1\linewidth]{"P/PR 6"}
\end{frame}

\begin{frame}{Implication}
	\begin{enumerate}
		\item In autocracies, class differences exist.\pause
		\begin{itemize}
			\item Urban working class is greater than urban middle class in universal welfare
			\item Class coalition matters in autocracies\pause
		\end{itemize}
		\item The relationship is affected by party institutionalization but still holds.\pause
		\begin{itemize}
			\item Institutionalized parties matter (consistent with recent works).\pause
		\end{itemize}
		\item Effect of coalitions are mediated by institutionalized parties.
	\end{enumerate}
\end{frame}

\section{Appendix}
\begin{frame}{Descriptive Statistics}
\begin{table}[ht]
\centering
\caption{Descriptive statistics}
	\begin{adjustbox}{width=0.7\textwidth}
\begin{tabular}{cccccc}
  \hline
		Variables & Obs. & Mean & Std. & Min. & Max. \\ 

  \hline
		Universal Welfare & 2,187 & 15.50 & 10.20 & 0.00 & 42.00 \\
		[0.5em] 
		Party Elites & 2,187 & 0.27 & 0.44 & 0.00 & 1.00 \\ 
		[0.5em]
		Military & 2,187 & 0.25 & 0.43 & 0.00 & 1.00 \\ 
		[0.5em]
		Urban Working & 2,187 & 0.06 & 0.23 & 0.00 & 1.00 \\ 
		[0.5em]
		Urban Middle & 2,187 & 0.03 & 0.16 & 0.00 & 1.00 \\ 
		[0.5em]
		Party Institutionalization. & 2,187 & 0.45 & 0.24 & 0.02 & 0.97 \\ 
		[0.5em]
		Logged Population & 2,187 & 9.00 & 1.46 & 5.55 & 14.01 \\ 
		[0.5em]
		Logged GDP per capita. & 2,187 & 7.98 & 0.86 & 4.90 & 11.65 \\ 
		[0.5em]
		Resource Dependency & 2,187 & 5.82 & 11.43 & 0.00 & 100.00 \\ 
		[0.5em]
		WWI & 2,187 & 0.02 & 0.12 & 0.00 & 1.00 \\ 
		[0.5em]
		WWII & 2,187 & 0.03 & 0.17 & 0.00 & 1.00 \\ 
		[0.5em]
		Cold War & 2,187 & 0.70 & 0.46 & 0.00 & 1.00 \\ 
		[0.5em]
   \hline
\end{tabular}
\end{adjustbox}
\label{tab:appendesc}
\end{table}
\end{frame}

\begin{frame}<beamer:0>
	\bibliographystyle{apsr}
	\bibliography{19AuthWelfare.bib}
\end{frame}
\end{document}
