
\documentclass{Bredelebeamer}
\usepackage[utf8]{inputenc}
\usepackage[T1]{fontenc}
\usepackage{scrextend}
\usepackage{booktabs}
\changefontsizes{10pt}
\usepackage{natbib}
\usepackage{adjustbox}
\usepackage{graphicx}
\newcommand\Fontvi{\fontsize{6}{7.2}\selectfont}
\bibpunct{(}{)}{;}{a}{}{,}
%%%%%%%%%%%%%%%%%%%%%%%%%%%%%%%%%%%%%%%%%%%%%%%%
\title{Why Do Authoritarian Regimes Provide Welfare?}
% Title

% Optional subtitle
\author{Sanghoon Park \inst{1}}
% The \ inst {...} command displays the speaker's affiliation.
% If there are several speakers: Marcel Dupont \ inst {1}, Roger Durand \ inst {2}
% Just add another institute to the model below.
\institute[University of South Carolina]
{
  \inst{1}%
  University of South Carolina\\Department of Political Science}
\date{\today}
% Optional. The date, usually the day of the conference
\subject{World Congress Presentation (6/26/2019)}
% It's used in PDF metadata

%%%%%%%%%%%%%%%%%%%%%%%%%%%%%%%%%%%%%%%%%%%%%%%%%%%%%%%%%%%%%%%%%%%%%
\begin{document}


\begin{frame}
  \titlepage
\end{frame}

\begin{frame}{Summary}
  \tableofcontents[hideallsubsections]
  % possibility of adding the option [pausesections]
\end{frame}

\section{Research Question}
\begin{frame}{Motivation}
	\begin{itemize}
		\item \textit{Major Topic}: Distribution and Redistribution
		\begin{itemize}
			\item Initial modern welfare state \pause $\rightarrow$ Germany of 1880s.
			\item Still, non-democracies provide welfare programs
		\end{itemize} \pause
		\item Existing Studies of welfare require fully democratic regime \pause
		\begin{itemize}
			\item Compensations \citep{Boix2001a,Adsera2002} 
			\item Power resources \citep{Huber1993,Bradley2003,Lupu2011,Lim2016} 
			\item Median Voters \citep{Meltzer1981,Iversen2006} 
		\end{itemize}\pause
		\item How can we explain why authoritarian regimes provide welfare?
	\end{itemize}	
\end{frame}

\section{Theory}
\begin{frame}{Not every authoritarian regime is the	same.}
\begin{itemize}
	\item In reality, we can observe authoritarian regimes have distinctive leadership groups \citep{Geddes2014}. 
	\item Generally, studies have a consensus that the party based regime has a larger
	coalition compared to the personalist regime, which has a small coalition to remain in office \citep{Levitsky2002,Levitsky2010,Gandhi2009}.
\end{itemize}
The welfare state literature + Empirical observations of authoritarian welfare provisions.
\begin{itemize}
	\item Given the fact that (i) authoritarian regimes can be seen as the diminished form of democracies and (ii) authoritarian regimes have variations,
	\item Some forms of authoritarian regimes are more likely to provide welfare.
\end{itemize}

\end{frame}

\begin{frame}{Class coalition, and cooptation}
Even though some autocrats use forces and coercions to rule the regime, not all do.
\begin{itemize}
	\item Two fundamental problems of autocrats, which cannot be solved by a single strategy \citep{Svolik2012}.
	\begin{enumerate}
		\item Problem of power-sharing, which is about challenges from the elites.
		\item Problem of control in which autocrats face threats from the masses they
		rule.
	\end{enumerate}
	\item Autocrats may have two strategies of coercion and cooptation \citep{Gandhi2007,Wright2008}
	\item Then, who is the target of cooptation? \pause $\rightarrow$  The critical supporting groups, which establishes the ruling coalition.
\end{itemize}
\end{frame}

\begin{frame}{Class coalition, and cooptation}
	\begin{itemize}
		\item How to identify the groups for ruling coalition?
		\begin{itemize}
			\item Party institutionalization \citep{Rasmussen2019}
			\begin{itemize}
				\item The more autocrats with institutionalized parties want to maintain power, the more likely they are to run universal social policies
			\end{itemize}
		\end{itemize}
		\item However, we cannot assume that parties in authoritarian regimes are identical to those in democracies.
		\begin{itemize}
			\item Not only party institutionalization, but the types of welfare can show the different class-coalition tendency across authoritarian regimes.
			\item The types of welfare delivery (cash benefits versus service benefits) would vary depending on the regime types \citep{Bambra2005}.
		\end{itemize}
	\end{itemize}
\end{frame}

\begin{frame}{Class coalition, and cooptation}
\begin{itemize}
  \item \textit{H$_1$}: When party institutionalization increases, welfare programs become universal \textit{de jure} (in terms of enacted and formal laws) and \textit{de facto} (in terms of the coverage of social groups).
  \item \textit{H$_2$}: Given party institutionalization, different authoritarian regimes increases different welfare programs closely related to their relevant class coalition \textit{de jure} (in terms of enacted and formal laws) and \textit{de facto} (in terms of the coverage of social groups).
\end{itemize}
\end{frame}

\section{Data and Empirical Specification}
\begin{frame}{Welfare Programs}
	\begin{itemize}
		\item DV: Welfare Encompassingness and Univeralsim
		\begin{itemize}
			\item Social Policy around the World (SPaW) Database \citep{Rasmussen2016}
			\item \textit{Encompassingness}: whether there is a major, national welfare law for each of the following risks.
			\item \textit{Universalism}: Range from 0–9, maximum 9-scores indicate all residents are automatically entitled to benefits, a fully universal system.
			\item The dataset covers 154 countries from 1790 to 2013.
		\end{itemize} 
	\end{itemize}
\end{frame}	

\begin{frame}{Welfare Programs}
\begin{figure}
	\centering
	\includegraphics[width=0.7\linewidth]{"../4. Figures/Figure0"}
	\caption{Global Trends of Welfare Programs by Regime Types}
	\label{fig:figure0}
\end{figure}
\end{frame}	

\begin{frame}{Class coalition cooptation}
	\begin{itemize}
		\item Distinct platforms (\textit{v2psplats}): how many parties among those with representation in the legislature have publicly available, and distinct, party platforms.
		\item Constituency linkages(\textit{v2psprlnks}): the most common form of linkage between parties and their constituents across all major parties.
		\item Particularistic versus public goods (\textit{v2dlencmps}, hereafter \texttt{Type of Goods}): the tendency of rent-seeking of autocrats regardless of the party linkage.
	\end{itemize}
\end{frame}
\begin{frame}
Authoritarian regime types from Autocratic Regimes dataset of \citet{Geddes2014}.
\begin{itemize}
	\item The interests represented in the leadership group help to explain and predict the behavior of authoritarian regime while they remain in power.
	\item Dominant-party, monarchy, military, and personal regimes covering 4,591 		observations from 1946 to 2010.
\end{itemize}
Also, several covariates to control the confounding influence on primary explanatory, and dependent variables are included:
\begin{itemize}
	\item \textit{Trade openness, the logged GDP per capita, the percentage of Elderly Population, the percentage of the Youth Population, urbanization, the size of military, resource dependence}
\end{itemize}

\end{frame}

\begin{frame}{Model Specification}
\begin{itemize}
	\item Data: unbalanced, pooled time-series cross-sectional data of class coalition cooptation variables, authoritarian regime types, and welfare programs
	\begin{itemize}
		\item Coverage: 74 autocracies, respectively, during the period 1816–2013. 
	\end{itemize}
	\item Ordinary least squares (OLS) estimator with panel-corrected standard errors.
	\begin{itemize}
		\item Time- and unit- fixed effect included.
		\item All explanatory variables are lagged by 3 years.
	\end{itemize}
\end{itemize} 
\end{frame}

\section{Empirical Analysis}
\begin{frame}{Party Institutionalization and Welfare}
\begin{figure}
	\centering
	\includegraphics[width=0.7\textwidth]{"../4. Figures/Coefplot1"}
	\caption{Party Institutionalization and Welfare Encompassingness}
	\label{fig:coefplot1}
\end{figure}
\end{frame}

\begin{frame}{Party Institutionalization and Welfare}
	\begin{figure}
		\centering
		\includegraphics[width=0.7\textwidth]{"../4. Figures/Coefplot2"}
		\caption{Party Institutionalization and Welfare Universalism}
		\label{fig:coefplot2}
	\end{figure}
\end{frame}

\begin{frame}{Disaggregated Universalisms}
\begin{figure}
	\centering
	\includegraphics[width=1\textwidth]{"../4. Figures/Coefplot3"}
	\caption{Party Institutionalization and Disaggregated Universalism}
	\label{fig:coefplot3}
\end{figure}
\end{frame}

\begin{frame}{Class coalition, Welfare Benefits}
\begin{figure}
	\centering
	\includegraphics[width=\linewidth]{"../4. Figures/margins1"}
	\caption{Marginsplot by regime type on Cash benefits}
	\label{fig:marginscash}
\end{figure}
\end{frame}

\begin{frame}{Class coalition, Welfare Benefits}
	\begin{figure}
		\centering
		\includegraphics[width=\linewidth]{"../4. Figures/margins2"}
		\caption{Marginsplot by regime type on non-Cash benefits}
		\label{fig:marginsnon}
	\end{figure}
\end{frame}

\section{Conclusion}
\begin{frame}{Conclusion}
	\begin{itemize}
		\item The first hypothesis is partially confirmed since
		the party institutionalization only increases the \textit{Encompassingness}, which means the coverage of risk-areas. 
		\begin{itemize}
			\item Why? $\rightarrow$ Disaggregate \textit{Universalism} by programs
			\item I find that party institutionalization does not
			have equal influence on the different welfare programs.
			\item There is no reason to assume that parties have inherent features to expand welfare programs.
		\end{itemize} 
		\item How about Class-coalition cooptation?
		\begin{itemize}
			\item Autocrats would want to deliver the resources they promised through the channel without autocrats’ private goods.
			rule. 
			\item Welfare programs are expected to be different depending on the groups on which the autocrat depends for support.
			\item The influence of party institutionalization 			conditional on different welfare programs depends on the class coalition—authoritarian regime types.
		\end{itemize}
	\end{itemize}	
\end{frame}

\begin{frame}<beamer:0>
	\bibliographystyle{apsr}
	\bibliography{19AuthWelfare.bib}
\end{frame}
\end{document}
