\documentclass{Bredelebeamer}
\usepackage[utf8]{inputenc}
\usepackage[T1]{fontenc}
\usepackage{scrextend}
\usepackage{booktabs}
\changefontsizes{11pt}
\usepackage{natbib}
\usepackage{adjustbox}
\usepackage{graphicx}

\bibpunct{(}{)}{;}{a}{}{,}
%%%%%%%%%%%%%%%%%%%%%%%%%%%%%%%%%%%%%%%%%%%%%%%%
\title{Why Do Authoritarian Regimes Provide Welfare Programs?}
% Title

% Optional subtitle
\author{Sanghoon Park \inst{1}}
% The \ inst {...} command displays the speaker's affiliation.
% If there are several speakers: Marcel Dupont \ inst {1}, Roger Durand \ inst {2}
% Just add another institute to the model below.
\institute[University of South Carolina]
{
  \inst{1}%
  University of South Carolina\\Department of Political Science}
\date{\today}
% Optional. The date, usually the day of the conference
\subject{World Congress Presentation (6/26/2019)}
% It's used in PDF metadata

%%%%%%%%%%%%%%%%%%%%%%%%%%%%%%%%%%%%%%%%%%%%%%%%%%%%%%%%%%%%%%%%%%%%%
\begin{document}


\begin{frame}
\titlepage
\end{frame}

\section{Research Question}
\begin{frame}[t]{Research Question}
	\begin{itemize}
		\item Existing studies of welfare assume fully democratic regime.\pause
		\begin{itemize}
			\item Compensations %\citep{Boix2001a,Adsera2002} 
			\item Median Voters %\citep{Meltzer1981,Iversen2006} 
			\item Power resources %\citep{Bradley2003,Korpi2006,Lupu2011} 
		\end{itemize}\pause
		\item How can we explain?
		\begin{itemize}
			\item why do authoritarian regimes provide welfare?
			\item are there variations of authoritarian welfare states?
		\end{itemize}
	\end{itemize}	
\end{frame}

\section{Theory}
\begin{frame}{Understanding of Welfare states}
\begin{alertblock}{Definition}
	\begin{itemize}
		\item A state in which organized power is deliberately used to modify the play of market forces.
		\item Minimum income, social insurance, and universal services
	\end{itemize}
\end{alertblock}
\begin{itemize}
	\item The structure of class coalitions presses govt. toward specific types of welfare state \citep{Gosta1990}.
	\item For example, red-green alliance of Sweden $\rightarrow$ universal %welfare programs.
\end{itemize}
\end{frame}

\begin{frame}{Welfare Regimes in Autocracies}
\begin{itemize}
	\item Leaders maintain coalitions of supporters by public, private goods \citep{BuenodeMesquita2003}.
	\item Autocracies have different coalitions \citep{Gandhi2009,Levitsky2010}.
	\item Autocrats also have incentives to provide public goods \citep{Wintrobe1998}
\end{itemize}
\vspace{0.3in}\pause
The problem is \textbf{\textit{who is the target?}}. %autocrats should care}}. \centering 
\end{frame}

\begin{frame}{Classes}
In authoritarian regimes,
	\begin{itemize}
		%\item the welfare might be the same as the means to co-opt the classes from selectorate for ruling coalitions.
		\item welfare = co-optation
		\item Extent of welfare depends on class. %the regime should co-opt.
	\end{itemize}
Assumption: individuals in similar strata have converging preferences over social policies.
	\begin{itemize}
		\item Income-based: \textbf{middle class} and \textbf{working class}
		\item Institutional-based: \textbf{party elite} and \textbf{military}
		\end{itemize}
\end{frame}

\begin{frame}{Classes}
\begin{enumerate}
	\item Middle class can be heterogeneous.
	\citep{Dahlum2019}. % group depending on whether it is based on rural or urban area. 
	\begin{itemize}
		\item Urban middle class $\rightarrow$ higher leverage %for autocracies as they have motivations and capability for regime transition.
	\end{itemize} 
	\item Working class
	\begin{itemize}
		\item Rural working class also affected by urbanization, industrialization.
		\item Elites have tried to co-opt both (e.g., the Russian Revolution).
	\end{itemize}
	\item The party elite and the military
	\begin{itemize}
		\item Specific institutions create class of elites with distinct interests, incentives.
	\end{itemize}
\end{enumerate}
\end{frame}

\begin{frame}{Class Coalitions}
\begin{itemize}
	\item Different classes $\rightarrow$ different welfare programs.
	\begin{itemize}
		\item Democratization literature emphasizes threats of middle class.
		\item Working class prefers more extensive welfare than middle.
		\begin{itemize}
			\item Working class more sensitive to change in distribution of wealth, welfare.% since it depends primarily on the labor force.
		\end{itemize}
	\end{itemize}\pause
	\item Hypotheses ($\text{\textit{H}}_1$): Working class has greater influence on universal welfare programs than the middle class.
\end{itemize}
\end{frame}

\begin{frame}{Alternative: Party Institutionalization}
\begin{itemize}
	\item Recent works focus on capacity of parties to provide public goods.
	\begin{itemize}
		\item Autocrats with institutionalized parties $\rightarrow$ universal welfare programs \citep{Rasmussen2019}.
		\item Institutionalized party helps distribute resources.
	\end{itemize}\pause
	\item Hypotheses ($\text{\textit{H}}_2$): Higher levels of party institutionalization increase welfare programs universality.
\end{itemize}
\end{frame}

\section{Data and Empirical Specification}
\begin{frame}{Sample Selection}
Authoritarian Regimes: \centering \pause
\begin{figure}[!htbt]
	\centering
	\includegraphics[width=0.85\linewidth]{"../3. Datasets_Codebooks/Figures/Plot1"}
	\caption{The distribution of Democracies and Autocracies}
	\label{fig:plot1}
\end{figure}
\end{frame}	

\begin{frame}{Sample Selection}
After the Russian Revolution of 1917: \centering \pause
\begin{figure}[!htbt]
	\centering
	\includegraphics[width=0.9\linewidth]{"../3. Datasets_Codebooks/Figures/Plot3"}
	\caption{Time trends of the numbers of states by classes}
	\label{fig:plot2}
\end{figure}
\end{frame}

\begin{frame}{Dataset and methods}
\begin{itemize}
	\item Coverage: 95 authoritarian states from 1917-2000 (unbalanced)
	\item Essential data source
	\begin{itemize}
		\item Social Policies around the World data set
		\item Variety of Democracies (V-Dem) data set
	\end{itemize}
	\item Variable
	\begin{itemize}
		\item DV: Welfare programs (social coverage and delivery)
		\item EV: Class coalitions (working, urban middle, party elites, and military)
		\item CV: Logged GDP per capita (V-Dem), logged of population, and resource dependence \citep{Miller2015}
	\end{itemize}
\end{itemize}
\end{frame}

\begin{frame}{Dataset and methods}
\begin{figure}[!htbt]
	\centering
	\includegraphics[width=1\linewidth]{"../3. Datasets_Codebooks/Figures/Plot2"}
	\caption{The distribution of universal indices by classes in Autocracies}
	\label{fig:plot3}
\end{figure}
\end{frame}

\begin{frame}{Empirical Analysis}
\begin{figure}[!htbt]
	\centering
	\includegraphics[width=1\linewidth]{"../3. Datasets_Codebooks/Figures/Table1"}
	\caption{Class coalitions, Party Institutionalization, and Welfare Universalism}
	\label{fig:plot4}
\end{figure}
\end{frame}

%begin{frame}{Empirical Analysis}
%\begin{figure}[!htbt]
%	\centering
%	\includegraphics[width=1\linewidth]{"../3. Datasets_Codebooks/Figures/Plot4"}
%	\caption{Disaggregated Party Institutionalization and Welfare Universalism (95\% CI)}
%	\label{fig:plot5}
%\end{figure}
%\end{frame}

\begin{frame}{Empirical Analysis}
\begin{figure}[!htbt]
	\centering
	\includegraphics[width=1\linewidth]{"../3. Datasets_Codebooks/Figures/Plot5"}
	\caption{Predicted Universal Welfare of Working Class by Party\\Institutionalization (95\% CI)}
	\label{fig:plot6}
\end{figure}
\end{frame}

\section{Conclusion}
\begin{frame}{Conclusion}
\begin{enumerate}
	\item In autocracies, class differences exist.\pause
	\begin{itemize}
		\item Working class is greater than middle class in universal welfare
		\item Class coalition matters in autocracies\pause
	\end{itemize}
	\item The relationship affected by party institutionalization.\pause
	\begin{itemize}
		\item It means when I include it, class differences disappear.
		\item Institutionalized parties matter (consistent with recent works).\pause
	\end{itemize}
	\item Effect of coalitions are mediated by parties that emerge to capture them in autocracies.
\end{enumerate}
\end{frame}

\section{Appendix}
\begin{frame}{Descriptive Statistics}
	\begin{table}[ht]
		\centering
		\begin{tabular}{lccccc}
			\hline
			Variables & Obs. & Mean & Std. & Min. & Max. \\ 
			\hline
			UI\_SPaW & 3,067 & 14.20 & 10.47 & 0.00 & 48.00 \\ 
			UI\_VDem & 6,285 & 0.01 & 1.42 & -3.42 & 3.41 \\ 
			Party Elites & 6,288 & 0.27 & 0.45 & 0.00 & 1.00 \\ 
			Military & 6,288 & 0.28 & 0.45 & 0.00 & 1.00 \\ 
			Working Class & 6,288 & 0.06 & 0.24 & 0.00 & 1.00 \\ 
			Urban Middle & 6,288 & 0.02 & 0.14 & 0.00 & 1.00 \\ 
			Party Inst. & 5,311 & 0.44 & 0.25 & 0.00 & 0.97 \\ 
			Ln.Pop. & 5,186 & 8.73 & 1.58 & 4.12 & 14.05 \\ 
			Ln.GDPpc. & 5,351 & 8.02 & 1.01 & 4.90 & 12.30 \\ 
			Res.Dep. & 5,199 & 6.60 & 13.93 & 0.00 & 100.00 \\ 
			\hline
		\end{tabular}
	\label{tab:tab1} \caption{Descriptive Statistics of Data}
	\end{table}
\end{frame}
\begin{frame}{Deaggreagted Party Institutionaliztion}
	\begin{figure}[!htbt]
		\centering
		\includegraphics[width=1\linewidth]{"../3. Datasets_Codebooks/Figures/Plot7"}
		\caption{Predicted Universal Welfare of Working Class by Deaggreagted Party\\Institutionaliztion (95\% CI)}
		\label{fig:plot7}
	\end{figure}
\end{frame}
\begin{frame}<beamer:0>
	\bibliographystyle{apsr}
	\bibliography{19AuthWelfare.bib}
\end{frame}
\end{document}
