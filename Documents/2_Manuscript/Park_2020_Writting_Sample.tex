\documentclass[12pt, letterpage, notitlepage]{article}
\usepackage[fleqn]{amsmath} % assumes amsmath package installed
\usepackage{setspace}
\doublespacing
\usepackage{geometry}
\usepackage{titling}
\usepackage{blindtext}
\geometry{margin=1in}
\usepackage{graphics} % for pdf, bitmapped graphics files
\usepackage{epsfig} % for postscript graphics files
\usepackage{mathptmx} % assumes new font selection scheme installed
\usepackage{times} % assumes new font selection scheme installed

\usepackage{amssymb}  % assumes amsmath package installed
\usepackage[affil-it]{authblk}
\usepackage{natbib}
\bibpunct{(}{)}{;}{a}{}{,}
\usepackage{hyperref}
\usepackage{bookmark}
\hypersetup{
	colorlinks   = true, %Colours links instead of ugly boxes
	urlcolor     = blue, %Colour for external hyperlinks
	linkcolor    = blue, %Colour of internal links
	citecolor   = blue %Colour of citations,
}
\usepackage{xcolor}
\usepackage{lettrine}
\usepackage{longtable}
\usepackage{afterpage}
\usepackage{booktabs}
\usepackage[utf8]{inputenc}
\usepackage[sc]{mathpazo}
\usepackage[T1]{fontenc}
\usepackage{adjustbox}
\usepackage{ntheorem}
\theoremseparator{:}
\newtheorem{hyp}{Hypothesis}
\usepackage[toc,title,page]{appendix}

\makeatletter % <=======================================================
\renewcommand\@seccntformat[1]{}
\renewcommand{\@makefntext}[1]{%
	\setlength{\parindent}{0pt}%
	\begin{list}{}{\setlength{\labelwidth}{6mm}% 1.5em <==================
			\setlength{\leftmargin}{\labelwidth}%
			\setlength{\labelsep}{3pt}%
			\setlength{\itemsep}{0pt}%
			\setlength{\parsep}{0pt}%
			\setlength{\topsep}{0pt}%
			\footnotesize}%
		\item[\@thefnmark\hfil]#1% @makefnmark
	\end{list}%
}

\makeatother % <========================================================


\title{\bf Why Do Authoritarian Regimes Provide Welfare?}
\author{Sanghoon Park\\The University of South Carolina}
	
\date{\today}


\begin{document}
%\begin{titlingpage}
	\maketitle

How successfully can we explain the variations of welfare provision among countries with the existing theories? Scholars have investigated the causes and consequences of welfare state variation for several decades, and it has become one of the major themes in comparative politics. Questions of the welfare state, however, have covered the variations of welfare within advanced industrial democracies \citep{Pierson1996, Pierson2000, Miller2015c}. Despite a rich literature on this topic in industrialized democracies, little research has investigated the mechanisms in authoritarian regimes. Do authoritarian regimes provide welfare to their citizens? Research on the welfare of democracies expects that democracies would provide greater stability and more effective public policies for citizens to foster their quality of life \citep{Gerring2012}. Authoritarian regimes also design social policies and provide a set of welfare programs, but it is not clear for whom the policies and programs are targeting, unlike democracies. Then, what would be a key mechanism to explain the variation of welfare in the non-democratic world?\footnote{The paper uses the terms referring to non-democracies, such as authoritarian regime, dictatorship, and autocracy interchangeably.}\par

This paper evaluates why authoritarian regimes design social policies to promote the welfare of citizens. Social welfare policies are coordinated outcomes that link the political and economic realms together. Many theoretical studies have proposed that democracy is somehow related to higher levels of welfare provisions \citep{Muller1988, Sirowy1990, Korpi1998} with several key concepts; the logic of industrialism \citep{Heclo1974, Wilensky1975, PrzeworskiandLimongi1993}, inherent characteristics of democratic institutions \citep{Muller1988, Pierson1996, Gerring2012}, compensating for the specific social groups \citep{Cameron1978, Rodrik1998, Burgoon2001a, Adsera2002} and distribution of political resources across classes within a society \citep{Gosta1989, Gosta1990, Korpi1998}. The central mechanism of conventional wisdom explains variations of welfare provision with social pressures for extensive welfare under democracies since democratic governments are more subject to demands from citizens in theory. However, very few studies have attempted to explain the variation of welfare with the mechanisms in existing theories for authoritarian welfare states.\par

I argue that the limited focus on welfare provision in democracies stems from several flawed assumptions about the ability of modern autocracies to create similar programs and the mechanisms behind it. Although autocrats are more autonomous decision-makers than democratic leaders, they rarely rule alone \citep{Gandhi2006, Frantz2014a}. As such, social policies in authoritarian regimes are driven by the autocrats’ need to create stable ruling coalitions. The welfare provided in authoritarian regimes should differ depending on the groups on which the autocrat depends for support. The ability to effectively decommodify welfare and provide assistance should also vary based on regime characteristics and resource availability. I examine this using data on authoritarian regimes and welfare programs, showing that authoritarian welfare provision depends on the ruling coalition in place.\par

I organize the remainder of the paper as follows. In the next two sections, I review the theoretical backgrounds of the welfare states. From the previous explanations of the welfare state, I develop the theoretical framework of our analysis. The following sections describe the dataset and empirical models I have constructed to test the hypotheses generated by this framework. The subsequent sections present and discuss our empirical results, and we conclude by addressing issues for further inquiry.

\section{Literature Reviews} 
\subsection{Understanding Welfare Regimes} 
Studies of welfare state have mostly been conducted in democratic countries with comparable and available data \citep{Jessop1984, Gosta1989, Gosta1990, Rudra2005, Huber2008, Hudson2009}. From the 19th century, as Western democracies expanded the universal franchise, the levels and coverage of social policies had expanded. Previous studies have provided empirical evidence that democracy has seen broader and deeper levels of welfare or social spending \citep{Lindert2005,Stasavage2005,HaggardKaufman2008}. They have made great efforts to find the regularity of social policies in democracies, and the strands of inquiry are so-called ``Welfare state'' studies.\par

The phrase welfare state was first used to describe Labour Britain after 1945, and the first usage of the phrase rarely defined what it is but just described it with democratic sentiments such as social rights \citep[9-10]{Briggs1961}. The label ‘welfare state’ incorporates not only Britain but also all modern industrialized countries, defining “a state in which organized power is deliberately used to modify the play of market forces in at least three directions---first, by guaranteeing individuals and families a minimum income irrespective of the market value of their work or their property; second, but narrowing the extent of insecurity by enabling individuals and families to meet certain ‘social contingencies’ (for example, sickness, old age, and unemployment) which lead otherwise to individual and family crises; and third, by ensuring that all citizens without distinction of status or class are offered the best standards available in relation to a certain agreed range of social services” \citep[14]{Briggs1961}. Based on the definition, several studies have investigated what makes differences in social expenditure across countries \citep{Cameron1978, Rodrik1998, Adsera2002}.\par

However, \citet{Gosta1990} argues that we should not focus on particular social policy \textit{per se}, but on how different countries arrive at their peculiar public-private sector mix \citep[2]{Gosta1990}. According to \citet{Gosta1990}, tracing the development of a welfare state with the size of social expenditure (or public spending) may be ahistorical and overlook the process. \citet{Gosta1990} focused on the two significant dimensions of welfare, decommodification, and stratification. On the one hand, decommodification is an indicator as to where the status of individuals in the market is. A state can intervene to secure the lost in the market through compensating for potential loss of job, income. Also, a state can provide general welfare for protecting citizens from sickness, injuries, unemployment. On the other hand, stratification is the status of individuals based on their class positions \citep{Gosta1990}.\par

The fundamental mechanism of the research is based on a class coalition. With qualitative research, \citet{Gosta1990} suggests that the structure of class coalitions is vital, pressing the ruling government toward specific types of welfare states. Thus, it is essential which classes go to make a coalition and whether the coalition successfully takes the ruling initiative of the state. It implies that the attempt of \citet{Gosta1990} to develop a typology of welfare states is based on power distribution in a society. He is, therefore, centrally concerned with how to explain the power structures underlying different welfare states \citep[88]{Kemeny1995}. In other words, the three types of the welfare state in \citet{Gosta1990}---Liberal, Conservative, and Social democracy represent macro-features of how a country forms the class coalition to design and provides welfare \citep[646]{Scruggs2008}.

Subsequent studies of \citet{Gosta1990} continued to categorize the welfare state based on different characteristics \citep[14]{Arcanjo2006}; coverage, citizenship right \citep{Ferrera1996}, relative size of social expenditure as a percentage of gross domestic products \citep{Bonoli1997}, or needs and contributions \citep{Korpi1998}. The main argument of \citet{Gosta1990} and following studies is that differential structuring of power relationships between classes is a political and policy-making process, making it possible to distinguish between different types of the welfare state. In other words, the thesis of \citet{Gosta1990} is much more than a descriptive typology but is ultimately founded on power relationships between social classes \citep[89]{Kemeny1995}.\par

\subsection{Welfare Regimes in Autocracies}

\citet{Pierson2000} and \citet{Scruggs2008a} criticize that the \textit{three worlds of welfare state} are only tested in limited democracies and not eligible to explain the variations of welfare states, which are on the boundary between the types. It is partially right. We cannot solve the puzzle of welfare state concerning authoritarian regimes with the existing theories on the welfare state since the previous arguments assume particular values of democratic institutions. When a state is an authoritarian regime, it does not guarantee compensation its citizens for the losses from the market. In authoritarian regimes, the power of the state is more likely personalized, and the rules are more likely to be established simply and efficiently using fiat and forces \citep[4]{Gerring2012}. However, I argue that the mechanism of \citet{Gosta1990}'s three worlds can be the key to solve the puzzles in authoritarian regimes with `class coalition.' \par

On one hand, the power resource theory suggesting that democracies are more likely to provide universal welfare because they rely on broader coalitions is nothing more than the selectorate theory \citep{BDM1999, BuenodeMesquita2003}. The selectorate theory states that leaders, facing challengers who wish to depose them, maintain their coalitions of supporters by taxing and spending in ways that allocate mixtures of public and private goods. The theory assumes each member of the selectorate (\textit{S}) who is responsible for making a leader has an equal probability of being in a  winning coalition. According to this logic, as the size of the winning coalition (\textit{W}) becomes smaller or the size of the selectorate becomes larger, challenges are less likely to need or use the support of any particular individual when forming their winning coalition. It is riskier to defect when the size of \textit{W} shrinks or the size of \textit{S} grows.\par 

On the other hand, studies on authoritarian regimes argue that we should consider the \textit{varieties of authoritarian regimes}, which means not every authoritarian regime is the same \citep{Geddes1999, Geddes2014, Cheibub2010, Wahman2013, Roller2013, Soest2015}. In reality, we can observe authoritarian regimes have distinctive leadership groups such as dominant party, military, monarchy, or personalized factions \citep{Geddes2014}. Generally, studies have a consensus that the party based regime has a larger coalition compared to the personalist regime, which has a small coalition to remain in office \citep{Levitsky2002, Levitsky2010a, Gandhi2009}. It is the point where the two lines of inquiry of selectorate theory and varieties of authoritarian regimes meet to explain the authoritarian welfare state.\par

The welfare state literature provides some case studies that authoritarian regimes provide welfare programs \citep{Tang2000, Kwon2005a, Orenstein2008, Bader2015, Ong2015, Morgenbesser2017}. Regardless of the regime types, political leaders seek to stay in office. The more typical a dictator, the more inherently unstable his power is. Although the leaders of democracy are regularly exposed to pressure from below through various institutions, the autocrats are under fewer constraints. It makes autocrats more powerful than democratic leaders, but also makes the feedback on policy impossible, resulting in a dilemma that fundamentally weakens his leadership. In other words, the autonomy of the dictator entails cost \citep[335]{Wintrobe1998}. It means that autocrats have incentives to provide public goods to reduce the risks from the dilemma. 

Here, there are two differences between democracies and authoritarian regimes for welfare provisions. First, the leaders under authoritarian regimes are less constrained to make decisions than the leaders under democracies because the political power of authoritarian regimes is more likely to be personlized or centralized on particular individual or groups. Second, unlike democracies, authoritarian regimes only need to be concerned with the citizens who are essential for staying in office. In other words, authoritarian regimes do not have incentives to care ``all citizens.'' Thus, in terms of providing a public good to stay in office, we would expect some authoritarian regimes to be subject to smaller dynamics discussed in the literature of democratization on the welfare regime and democracies. Also, some forms of authoritarian regimes are expected to be more likely to provide welfare. In other words, autocrats are more autonomous decision-makers than democratic leaders; however, they do not rule alone. The needs of autocrats to secure their ruling coalition which is a coordinated social group to secure the political survival of autocrats and their stable ruling drive the social policies to promote welfare even in authoritarian regimes \citep{Svolik2012}. Thus, with smaller coalitions, authoritarian regimes provide welfare because it is closely related to the credible commitment for co-optation.

Explanations of welfare provision in democracies have something in common. They argue that the social groups belonging to selectorate transform their needs into a sort of pressure to push towards particular sets of social policies. For example, a simple but essential analytical model, relating democratic institutions and welfare, comes from \citet{Meltzer1981}, who focus on the effects of electoral competition.\footnote{\citet{Boix2003} provide overviews of this issue} This model assumes that the median voter is the critical voter to determine the size of government, which is measured by the share of income redistributed. It implies that the size of the government depends on the relationship between mean income and median income. Here, the size of selectorate that depends on the regime types is important. Whereas democracy provides a universal franchise, which leads the median voter to have a below-average income in an unequal society, one of the poor who favors higher taxes and more welfare. In authoritarian regimes, the median voter who can affect the policy-making process would belong to a much smaller selectorate. It means that the median voter in an authoritarian regime is more likely to be one of the rich or the upper-class \citep[2]{Yi2014}.\par

The studies on \textit{varieties of authoritarian regimes} have focused on which social groups matter and affect the decision of autocrats. From the \citet{Geddes1999} and \citet{Geddes2014}, scholars have attempted to uncover the decisive leadership group, which controls the policy-making process within the regime. However, when we connect the previous discussion of authoritarian welfare to class coalition---ruling coalition of autocracies, it becomes clear that we need to concern ourselves with how much autocrats consider the selectorate and how large the coalition size is because it shows the extent of welfare the autocrats should provide. It means that the welfare provided in authoritarian regimes should differ depending on the groups on which the autocrat depends for support.\par

However, existing approaches of discrete authoritarian regime types are not free from the validity issues \citep{Wilson2014}. \citet[52]{Roller2013} shows whether and how the choice of data sets impacts the empirical results on the determinants and consequences of authoritarian regimes. Also, most of the studies to examine performances of authoritarian regimes utilizes the method of residues of which we make all unidentified regime types as one type. It makes it challenging to state which authoritarian type is different from what.

\subsection{The Primacy of Coalitions}

Autocrats have two fundamental problems, which cannot be solved by a single strategy. One is the problem of power-sharing, which is about challenges from the elites. The other is the problem of control in which autocrats face threats from the masses they rule \citep{Svolik2012}. The processes to resolve the problems in authoritarian regimes are not transparent and formal, contrary to in democracies. When an autocrat faces these two fundamental problems, she may have two strategies of coercion and co-optation. The autocratic welfare is closely related to the latter strategy. For instance, authoritarian regimes establish institutions to encourage regime stability for autocrat through signaling secured property rights and credible commitment \citep{Gandhi2007, Wright2008}.

In authoritarian regimes, we can expect the welfare might be the same as the means to co-opt the classes from selectorate of which the members are potential constituents for the ruling coalition. Given that authoritarian social policies are to effectively stabilize and sustain the system through co-opting the supports of selectorate, social policies should be more credible and reliable than private goods, which the autocrat can distribute at his discretion. \citet{Knutsen2018} emphasizes that pensions are a good example of institutionalized club goods for autocrats to use with particular target groups. They specify old-age pensions as a useful tool for autocrats to reduce the threats of revolt and to co-opt their opponents. They argue that especially old-age pensions affect a wide range of beneficiaries as the pensions do not depend on an unexpected event such as unemployment, sickness. The old-age pensions may be relatively universal to other types of welfare. In the authoritarian regime, as mentioned above, autocrats only have the incentive to provide old-age pension programs to secure targeted supporter groups \citep[670]{Knutsen2018}. In conclusion, \citet[688]{Knutsen2018} argue that ``major pension programs are not restricted to democracies $\cdots$ Autocrats use pensions to direct resources to critical supporting groups.''

How universal welfare programs are offered, and which ones to focus more on, depends on which class the regime should co-opt. This study understands that class exerts social pressure on decision-makers through different and distinctive interests or means of political mobilization. Therefore, it is necessary to examine which classes are politically significant and identifiable \citep{Bean1998}. In this paper, I take the assumption of \citet[1495]{Dahlum2019}: individuals who belong to similar socioeconomic strata are more likely to have converging preferences over social policies.

%In Western Europe, agriculture was the dominant production structure of society until the industrial revolution between the late 18th and early 19th centuries emerged. The source of the political resources of the elite was based on agriculture. Although the landowners of the peasants and bourgeois expanded over time, the feudal manor lords occupied the land ownership steadily. The feudal lords---the agrarian elite has accumulated surplus labor or surplus products from the peasants or tenant farmers in the form of rents by lending common lands. \citet{Ansell2015a} suggest that the ruling elite in the early stage can be divided into two groups. They call the two groups as the relative economic elites and the autocratic elite \footnote{This study defines the relative economic elites as groups of people who have industrial goods (economic power) without political power.}. Also, \citet{Ansell2015a} considers the autocratic elite as the incumbent elite who gains his resources from its local agricultural section.

A line of inquiry argues economic and class-based explanations to account for democratic transition, development \citep[4-5]{ziblatt2017conservative}, and welfare state \citep{Huber1993,HuberStephens2001,Korpi2006}. The explanations usually classify classes that are important in a regime into two types, \textit{middle class} and \textit{working class}. On the one hand, the middle class is the group that has the economic capability. The middle class contributes to economic balance and, at the same time, serves as a buffer against political extremes. Socially, it also serves as a role model for low-income groups, while playing a role of social balance by having a specific collective identity that is distinct from the upper classes. For example, Some evaluate the middle class as the driving force for democratization historically \citep{Moore1993Social}. In line with literature on democratization, we can expect that the middle class can be seen as an effective source of threatening the regime. It also makes the middle class important when I explore the influence of classes on welfare programs in authoritarian regimes.

The middle class can be identified as a politically significant social group to co-opt since whether to make a coalition with the middle class determine the costs of repression. Also, theories that find the cause of economic development in the distribution of political power also suggested that the middle-class influences market growth by contributing to the development of democracy and preventing dictatorship \citep{Easterly2001}. Many studies suggest that middle-class supplies crucial support for the popular mobilizations \citep{Rosenfeld2017, Dahlum2019}. Regarding incentives to co-opt, however, the middle class can be seen as a heterogeneous group. Remaining in office, authoritarian leaders should make a coalition with the group that might have motivation and capability for regime transition \citep{Dahlum2019}. Thus, I expect that the middle class, in particular, the urban middle class, would have higher leverage for authoritarian leaders.

On the other hand, the working class can be an important factor in the development of redistributive social policy, partly through its effects on the partisan composition of government (\citealp{HuberStephens2001}, \citealp[794]{Rasmussen2018}). According to \citet[175]{Korpi2006}, the working class is a collectively mobilized group of employees who are dependent on the labor force. Mobilization of the working class is closely related to the development of the welfare state \citep{Shalev1983}. The expectation that the working class will have a converging preference is based on the assumption that the group of workers has the same interest in ensuring security from the risk of the market. \citet[1495]{Dahlum2019} argues that the urban working class (e.g., industrial workers) and the rural working class (peasants) show different levels of capabilities to organize nationwide or localize strikes for the regime transitions. However, in terms of co-optation, the two different working classes can have another different narratives of motivation; do they pursue same welfare programs? Although the urban working class is in the capitalist mode of production and has motivation to change their labor environment fundamentally, the rural working class is understood as the group in the feudal mode of production. Thus, when the treats of turmoil are deduced by the leverages and motivations of social groups, the urban working class would be the social class autocrats should co-opt.

%Since the rural working class can be influenced by urbanization or industrialization, the rural working class, and the urban class are not exclusive to each other. \citet[1932]{Chandra2002} shows an example of the Russian revolution. The rural working class is understood as the transitional group between the remnants of feudal and the capitalist mode of production. The rural working class and the urban working class were co-opted because they were in a continuous line by looking at the production of rural working classes as the basis of capitalist development. 

%Although \citet[1495]{Dahlum2019} argues that the urban working class (e.g., industrial workers) and the rural working class (peasants) show different levels of capabilities to organize nationwide or localize strikes for the regime transitions, it has different narratives in terms of co-optation. Since the rural working class can be influenced by urbanization or industrialization, the rural working class, and the urban class are not exclusive to each other. \citet[1932]{Chandra2002} shows an example of the Russian revolution. The rural working class is understood as the transitional group between the remnants of feudal and the capitalist mode of production. The rural working class and the urban working class were co-opted because they were in a continuous line by looking at the production of rural working classes as the basis of capitalist development. 

Besides the income based classes, groups based on specific institutions also matter with regards to the political survival of autocrats. \citet{Geddes1999} and \citet{Geddes2014} proposes different types of leadership groups are able to make the most important decisions in the authoritarian regime. This group makes key policies, and autocrats must retain the support of its members to remain in power, even though autocrats may also have substantial ability to influence the group's membership. Institution-based class refers to additional coalition members, and it provides valuable alternative bluntly constructed regime types. Specific institutions create a class of elites with distinct interests and incentives. For instance, parties are not the only way to mobilize the masses, but if created, it may offer goods to its leadership, party elites. 

The military and party elites are the institution-based groups, which show close ties with authoritarian leaders. First, the military does not depend on military leaders’ personal characteristics. Instead, the military is characterized by collegial forms meaning ``rule by the military as institution'' \citep{Geddes2014b, Kim2017}. I expect the military as the collegial group would have converging interests and capabilities over social policies. Secondly, when an authoritarian regime has party institutions, the party institution can serve interactive functions rather than one-way. It implies that party institution is not only the platform to mobilize the support from the masses \citep{Gandhi2006}, but also the mechanisms making the party elite seek particular interests. Through the party institution, the party elite can shape his interests, which are different from the interests of other classes.

\newpage
\bibliographystyle{apsr}
\bibliography{19AuthWelfare}


\end{document}